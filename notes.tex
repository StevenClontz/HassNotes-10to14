\documentclass[12pt]{article}

\usepackage{fancyhdr}

\usepackage{graphicx}

\pdfpagewidth 8.5in
\pdfpageheight 11in

\setlength\topmargin{0in}
\setlength\headheight{0in}
\setlength\headsep{0.2in}
\setlength\textheight{8in}
\setlength\textwidth{6in}
\setlength\oddsidemargin{0in}
\setlength\evensidemargin{0in}
\setlength\parindent{0in}
\setlength\parskip{0.1in} 

\pagestyle{fancy}
\headheight 35pt

\lhead{Steven Clontz}
\chead{Cal III (MATH 2630) Lecture Notes}
\rhead{Page \thepage}

\lfoot{\footnotesize Spring 2013}
\cfoot{}
\rfoot{\footnotesize Last saved on \today}
 
\usepackage{amssymb}
\usepackage{amsfonts}
\usepackage{amsmath}
\usepackage{mathtools}
\usepackage{amsthm}
\usepackage{wasysym} % for \smiley
      
\newcommand{\ds}{\displaystyle}
\newcommand{\vect}[1]{\mathbf{#1}}
\newcommand{\veci}{\mathbf{i}}
\newcommand{\vecj}{\mathbf{j}}
\newcommand{\veck}{\mathbf{k}}
\newcommand{\dvar}[1]{\,d{#1}}
\renewcommand{\div}{\textrm{div}\,}
\newcommand{\spin}{\textrm{spin}\,}
\newcommand{\proj}{\textrm{proj}}
\newcommand{\<}{\left<}
\renewcommand{\>}{\right>}


\begin{document}

\centerline{\bf Sections 10.1 - 10.2 Overview }

\begin{itemize}
\item Three-Dimensional Coordinates (10.1)

	\begin{itemize}
	\item Distance between points in 3D space
		\[D = \sqrt{(x_2 - x_1)^2 + (y_2 - y_1)^2 + (z_2 - z_1)^2}\]
		
	\item Simple planes in 3D Space
		\[x=a,\, y=b,\, z=c\]
	
	\item Spheres in 3D Space
		\[(x-x_0)^2 + (y-y_0)^2 + (z-z_0)^2 = a^2\]
	\end{itemize}

\item Vectors (10.2)
	\begin{itemize}
	\item Definition of a Vector
	
		\begin{itemize}
		\item A vector $\vect{v}=\overrightarrow{v}$ is a mathematical object which stores length (magnitude) and direction, and can be thought of as a directed line segment.
	
		\item Two vectors with the same length and direction are considered equal, even if they aren't in the same position. 

		\item We often (but not always) assume the initial point (the one without an arrow) lays at the origin.
		\end{itemize}
		
	\item Component Form
	
		$\<v_x,v_y,v_z\>$ is equal to the vector with initial point at $(0,0,0)$ and terminal point at $(v_x,v_y,v_z)$.
		
	\item 2D vs 3D Vectors
	
		\[\<a,b\>=\<a,b,0\>\]
		
	\item Position Vector
	
		If $P=(a,b,c)$ is a point, then $\vect{P}=\<a,b,c\>$ is its \textbf{position vector}. 

		We assume $(a,b,c)=\<a,b,c\>$.

	\item Vector Between Points
	
	The vector from $P_1 = (x_1,y_1,z_1)$ to $P_2 = (x_2,y_2,z_2)$ is \[\vect{P_1P_2} = \overrightarrow{P_1P_2} = \<x_2-x_1,y_2-y_1,z_2-z_1\>\]
	
	\item Length of a Vector
	
		\[|\vect{v}| = |\<v_1,v_2,v_3\>| = \sqrt{v_1^2 + v_2^2 + v_3^2}\]
	
	\item The Zero Vector
	
	\[\vect{0} = \overrightarrow{0} = \<0,0,0\>\]
	
	\item Vector Operations
		\begin{itemize}
		\item Addition
			\[\<v_1,v_2,v_3\> + \<u_1,u_2,u_3\> = \<v_1+u_1,v_2+u_2,v_3+u_3\>\]
		\item Scalar Multiplication
			\[k\<v_1,v_2,v_3\> = \<kv_1,kv_2,kv_3\> \]
		\end{itemize}

	\item Vector Operation Properties
		\begin{enumerate}
		\item $\vect{u}+\vect{v} = \vect{v}+\vect{u}$
		\item $(\vect{u}+\vect{v})+\vect{w} = \vect{u}+(\vect{v}+\vect{w})$
		\item $\vect{u}+\vect{0} = \vect{u}$
		\item $\vect{u}+(-\vect{u}) = \vect{0}$
		\item $0\vect{u} = \vect{0}$
		\item $1\vect{u} = \vect{u}$
		\item $a(b\vect{u}) = (ab)\vect{u}$
		\item $a(\vect{u} + \vect{v}) = a\vect{u} + a\vect{v}$
		\item $(a+b)\vect{u} = a\vect{u} + b\vect{u}$
		\end{enumerate}

	\item Unit Vectors
		\begin{itemize}
		\item A \textbf{unit vector} or \textbf{direction} is any vector whose length is $1$.
		
		\item Standard unit vectors
			\begin{itemize}
			\item $\veci = \<1,0,0\>$
			\item $\vecj = \<0,1,0\>$
			\item $\veck = \<0,0,1\>$
			\end{itemize}

		\item Standard Unit Vector Form:
			\[\<v_x,v_y,v_z\> = v_x\veci + v_y\vecj + v_z\veck\]

		\item Length-Direction Form:
			\[\vect{v} = |\vect{v}|\frac{\vect{v}}{|\vect{v}|}\]

		\end{itemize}
	\end{itemize}
\end{itemize}

\newpage
	
\centerline{\bf 10.3 The Dot Product}
	
		\begin{itemize}
		\item Dot Product
			\[ \vect{u} \cdot \vect{v} = \<u_1,u_2,u_3\>\cdot\<v_1,v_2,v_3\> = u_1v_1 + u_2v_2 + u_3v_3 \]
	
		\item Angle between vectors
			\[\cos\theta = \frac{\vect{u}\cdot\vect{v}}{|\vect{u}||\vect{v}|}\] 
			
		\item Alternate Dot Product formula
			\[\vect{u} \cdot \vect{v} = |\vect{u}||\vect{v}|\cos \theta \]
			
		\item Orthogonal Vectors
			\begin{itemize}
			\item	$\vect{u},\vect{v}$ are orthogonal if $\vect{u} \cdot \vect{v} = 0$
			\item $\vect{u},\vect{v}$ are orthogonal if the angle between them is $\frac{\pi}{2} = 90^\circ$
			\item $\vect{0}$ is orthogonal to every vector
			\end{itemize}
		
		\item Dot Product Properties
			\begin{enumerate}
			\item $\vect{u} \cdot \vect{v} = \vect{v}\cdot\vect{u}$
			\item $(c\vect{u})\cdot \vect{v} = \vect{u} \cdot (c\vect{v}) = c(\vect{u} \cdot \vect{v})$
			\item $\vect{u} \cdot (\vect{v} + \vect{w}) = \vect{u}\cdot\vect{v} + \vect{u}\cdot \vect{w}$
			\item $\vect{u} \cdot \vect{u} = |\vect{u}|^2$
			\item $\vect{0} \cdot \vect{u} = 0$
			\end{enumerate}
		
		\item Projection Vector
			\[\proj_{\vect{v}}(\vect{u}) = \left(\frac{\vect{u}\cdot\vect{v}}{|\vect{v}|} \right)\frac{\vect{v}}{|\vect{v}|} = \left(\frac{\vect{u}\cdot\vect{v}}{|\vect{v}|^2} \right)\vect{v}\]

		\item Work
			\[W = \vect{F} \cdot \vect{D} = |\vect{F}||\vect{D}|\cos \theta\]\
		
		\item \textbf{Suggested Exercises for 10.3}
			\begin{itemize}
			\item Finding and applying dot products: 1-8
			\item Work done by a constant vector force: 39-40
			\end{itemize}
		\end{itemize}
	
\newpage

\centerline{\bf 10.4 The Cross Product}
	
		\begin{itemize}
		\item Right-hand rule

			Any method for determining a special orthogonal direction used throughout mathematics and physics, with respect to an ordered pair of vectors $\vect{u},\vect{v}$
			
		\item Unit Normal Vector

		  The vector $\vect{n}$ orthogonal to an ordered pair of vectors $\vect{u},\vect{v}$ following the right-hand rule
		
		\item Cross Product
		 	\[\vect{u} \times \vect{v} = (|\vect{u}||\vect{v}|\sin \theta) \vect{n} \]
		
		\item Parallel Vectors
			\begin{itemize}
			\item	$\vect{u},\vect{v}$ are parallel if $\vect{u} \times \vect{v} = 0$
			\item $\vect{u},\vect{v}$ are parallel if the angle between them is $0=0^\circ$ or $\pi = 180^\circ$
			\item $\vect{0}$ is parallel to every vector
			\end{itemize}
		
		\item Cross Product Properties
		
			\begin{enumerate}
			\item $(r\vect{u}) \times (s\vect{v}) = (rs)(\vect{u} \times \vect{v})$
			\item $\vect{u} \times (\vect{v} + \vect{w}) = \vect{u} \times \vect{v} + \vect{u} \times \vect{w}$
			\item $(\vect{v} + \vect{w}) \times \vect{u} = \vect{v} \times \vect{u} + \vect{w} \times \vect{u}$
			\item $\vect{v} \times \vect{u} = -(\vect{u} \times \vect{v})$
			\item $\vect{0} \times \vect{u} = \vect{0}$
			\end{enumerate}

		\item Standard Unit Vector Cross Products
			\begin{enumerate}
			\item $\veci \times \vecj = \veck$
			\item $\vecj \times \veck = \veci$
			\item $\veck \times \veci = \vecj$
			\end{enumerate}

		\item Parallelogram Area
			The area of a parallelogram determined by $\vect{u},\vect{v}$ is \[|\vect{u} \times \vect{v}| = |\vect{u}||\vect{v}|\sin \theta\]
		
		\newpage
		\item Determinants
			\begin{itemize}
			\item 2x2 Determinant
			
			\[
\begin{array}{|c c|}
a & b \\
c & d \\
\end{array}
			= ad - bc
			\]
			
			\item 3x3 Determinant

\begin{center}\begin{tabular}{rl}
	$
		\begin{array}{|c c c|}
		a_1 & a_2 & a_3 \\
		b_1 & b_2 & b_3 \\
		c_1 & c_2 & c_3 \\
		\end{array}
	$
	&
	$
			= a_1 \,
		\begin{array}{|c c|}
		b_2 & b_3 \\
		c_2 & c_3 \\
		\end{array}
			- a_2 \,
		\begin{array}{|c c|}
		b_1 & b_3 \\
		c_1 & c_3 \\
		\end{array}
			+ a_3 \,
		\begin{array}{|c c|}
		b_1 & b_2 \\
		c_1 & c_2 \\
		\end{array}
	$
	\\ [1ex] \\ &
	$
		 = a_1 \,
		\begin{array}{|c c|}
		b_2 & b_3 \\
		c_2 & c_3 \\
		\end{array}
			+ a_2 \,
		\begin{array}{|c c|}
		b_3 & b_1 \\
		c_3 & c_1 \\
		\end{array}
			+ a_3 \,
		\begin{array}{|c c|}
		b_1 & b_2 \\
		c_1 & c_2 \\
		\end{array}
	$
	\\ [1ex] \\ &
	$= (a_1b_2c_3 + a_2b_3c_1 + a_3b_1c_2) - (a_3b_2c_1 + a_1b_3c_2 + a_2b_1c_3)$
\end{tabular}\end{center}
			
			\end{itemize}
		\item Computing Cross Products
\[
		\vect{u} \times \vect{v} = 
\begin{array}{|c c c|}
\veci & \vecj & \veck \\
u_1 & u_2 & u_3 \\
v_1 & v_2 & v_3 \\
\end{array}
		=
		\<
\begin{array}{|c c|}
u_2 & u_3 \\
v_2 & v_3 \\
\end{array}
		\,,\,
\begin{array}{|c c|}
u_3 & u_1 \\
v_3 & v_1 \\
\end{array}
		\,,\,
\begin{array}{|c c|}
u_1 & u_2 \\
v_1 & v_2 \\
\end{array}
		\>
\]
		\[
		=
		\<u_2v_3-u_3v_2\,,\,u_3v_1-u_1v_3\,,\,u_1v_2-u_2v_1\>
		\]
		
		Shortcut ``long multiplication'' method:
		
		\[
\begin{array}{rcccccl}
\langle& u_1 & , & u_2 & , & u_3 & \rangle \\
\times\langle & v_1 & , & v_2 & , & v_3 & \rangle \\\hline
\langle & u_2v_3-u_3v_2 & , & u_3v_1-u_1v_3 & , & u_1v_2-u_2v_1 & \rangle
\end{array}
		\]
		
		\item Torque
		
		\[\overrightarrow{\tau} = \vect{r} \times \vect{F} = (|\vect{r}||\vect{F}|\sin \theta)\vect{n} \]
		
		\item Triple Scalar (or ``Box'') Product
		
		\[
		(\vect{u}\times\vect{v})\cdot\vect{w} =
		\begin{array}{|c c c|}
		u_1 & u_2 & u_3 \\
		v_1 & v_2 & v_3 \\
		w_1 & w_2 & w_3 \\
		\end{array}
		\]
		
		Its absolute value $|(\vect{u}\times\vect{v})\cdot\vect{w}|$ gives the volume of a parallelpiped determined by the three vectors.
		
		\item \textbf{Suggested Exercises for 10.4}
			\begin{itemize}
			\item Finding cross products: 1-14
			\item Finding areas and unit normal vectors using cross products: 15-18
			\item Finding volumes using cross products: 19-22
			\item Computing torque: 25-26
			\end{itemize}
		\end{itemize}

\newpage

\centerline{\bf 10.5 Lines and Planes in Space}
	
		\begin{itemize}
		\item Vector Equation for a Line
			\[\vect{r}(t) = \vect{P_0} + t\vect{v}\] for $-\infty < t < \infty$
		
		\item Parametric Equations for a Line
			\[x = x_0 + tv_1, y = y_0 + tv_2, z = z_0 + tv_3\] for $-\infty < t < \infty$
		
		\item Line Passing through a pair of points
			\[\vect{r}(t) = \vect{P_0} + t(\vect{P_0P_1})=(1-t)\vect{P_0}+t\vect{P_1}\] for $-\infty < t < \infty$
		
		\item Line Segment joining a pair of points
			\[\vect{r}(t) = \vect{P_0} + t(\vect{P_0P_1})=(1-t)\vect{P_0}+t\vect{P_1}\] for $0 \leq t \leq 1$
		
		\item Distance from a Point to a Line
			\[d = \frac{|\vect{PS} \times \vect{v}|}{|\vect{v}|}\]
		
		\item Equation for a Plane
			\[\vect{n} \cdot (\vect{P_0}\vect{P}) = 0\] 
			\[A(x-x_0) + B(y-y_0) + C(z-z_0) = 0\] 
		
		\item Line of Intersection of Two Planes
			\[\vect{r}(t) = \vect{P} + t(\vect{n_1} \times \vect{n_2})\]
		
		\item Distance from a Point to a Plane
			\[d = \frac{|\vect{PS} \cdot \vect{n}|}{|\vect{n}|}\]
\newpage
		\item \textbf{Suggested Exercises for 10.5}
			\begin{itemize}
			\item Finding parametric equations for lines: 1-12
			\item Finding parametrizations for line segments: 13-20
			\item Finding equations for planes: 21-26
			\item Distance from a point to a line: 33-38
			\item Distance from a point to a plane: 39-44
			\end{itemize}
		
		\end{itemize}


\newpage

\centerline{\bf 10.6 Cylinders and Quadratic Surfaces }

	\begin{itemize}
	\item Sketching surfaces

		\begin{itemize}
		\item To sketch a 3D surface, sketch planar cross-sections
			\begin{itemize}
			\item $z=c$ is parallel to $xy$ plane
			\item $y=b$ is parallel to $xz$ plane
			\item $x=a$ is parallel to $yz$ plane
			\end{itemize}
		\end{itemize}

	\item Cylinders

		\begin{itemize}
		\item A \textbf{cylinder} is any surface generated by moving a planar along a line normal to that plane.
		\item A 3D surface defined by a function of only two variables results in a cylinder.
		\end{itemize}

	\item Quadric Surfaces

		\begin{itemize}
		\item A \textbf{quadric surface} is any surface defined by a second degree equation of $x,y,z$: \[Ax^2+Bx+Cy^2+Dy+Ez^2+Fz=G\]
		\item Most helpful to consider the cross-sections in each of the coordinate planes.
		\end{itemize}

	\item Ellipsoids
		\begin{itemize}
		\item Cross-sections in the coordinate planes include
			\begin{itemize}
			\item Three ellipses
			\end{itemize}
		\end{itemize}

	\item Elliptical Cone
		\begin{itemize}
		\item Cross-sections in the coordinate planes include
			\begin{itemize}
			\item Two double-lines
			\item One point (with parallel ellipses)
			\end{itemize}
		\item Cross-sections parallel to the point cross-section are ellipses.
		\end{itemize}

	\newpage

	\item Elliptical Paraboloid
		\begin{itemize}
		\item Cross-sections in the coordinate planes include
			\begin{itemize}
			\item Two parabolas
			\item One point (with parallel ellipses)
			\end{itemize}
		\item Cross-sections parallel to the point cross-section are ellipses.
		\end{itemize}

	\item Hyperbolic Paraboloid
		\begin{itemize}
		\item Cross-sections in the coordinate planes include
			\begin{itemize}
			\item Two parabolas (with parallel parabolas)
			\item One double line (with parallel hyperbolas)
			\end{itemize}
		\end{itemize}

	\item Hyperboloid of One Sheet
		\begin{itemize}
		\item Cross-sections in the coordinate planes include
			\begin{itemize}
			\item Two hyperbolas
			\item One ellipsis (with parallel hyperbolas)
			\end{itemize}
		\end{itemize}

	\item Hyperboloid of Two Sheets
		\begin{itemize}
		\item Cross-sections in the coordinate planes include
			\begin{itemize}
			\item Two hyperbola
			\item One empty cross-section (with parallel hyperbolas)
			\end{itemize}
		\item Cross-sections parallel to the empty cross-section are ellipses.
		\end{itemize}

	\item \textbf{Suggested Exercises for 10.6}
		\begin{itemize}
		\item Identify surfaces from equations: 1-12
		\item Sketching surfaces: 13-44
		\end{itemize}
	\end{itemize}

% \newpage
	
% 	\centerline{\bf 11.1 Vector Functions and their Derivatives}
	
% 	In Calculus I we first learned how to find the rate of change of a function, its derivative. In this section, we'll extend that idea to vectors.
	
% 	\begin{itemize}
% 	\item Curves, Paths, and Vector Functions (pg. 663)
% 		\begin{itemize}
% 		\item The movement of a particle in space can be thought of as a \textbf{path} or \textbf{curve}.
		
% 		\item A function which maps a moment in time to a position on a path is known as a \textbf{position function}. It can be defined with \textbf{parametric equations} as \[x=x(t)\] \[y=y(t)\] \[z=z(t)\] or with a parametrizing \textbf{vector function} \[\vect{r}(t) = \<x(t),y(t),z(t)\> = x(t)\veci + y(t)\vecj + z(t)\veck\]
		
% 		\item The $x(t),y(t),z(t)$ functions are known as \textbf{component functions} of the vector function $\vect{r}(t)$.
% 		\end{itemize}
% 	\item Vector Function Limits (pg. 665)
% 		\begin{itemize}
% 		\item If the value of the vector function $\vect{r}(t)$ becomes arbitrarily close to the vector $\vect{L}$ as values of $t$ close to $t_0$ are plugged into the function, then the \textbf{limit of $\vect{r}(t)$ as $t$ approaches $t_0$} is $\vect{L}$, or \[\lim_{t\to t_0} \vect{r}(t) = \vect{L}\]
% 			\begin{itemize}
% 			\item Precise definition: \newline
% 			Suppose there exists a function $\delta(\epsilon)$ so that for all positive numbers $\epsilon>0$ and all numbers $t$ where $|\vect{r}(t)-\vect{L}|<\epsilon$, it follows that $|t-t_0|<\delta(\epsilon)$. Then we say that $\lim_{t\to t_0} \vect{r}(t) = \vect{L}$.
% 			\end{itemize}
% 		\item If $\vect{r}(t) = \<f(t),g(t),h(t)\>$, then it follows that \[\lim_{t\to t_0} \vect{r}(t) = \<\lim_{t\to t_0} f(t), \lim_{t\to t_0} g(t), \lim_{t\to t_0} h(t)\>\]
% 		\end{itemize}
% 	\newpage
% 	\item Continuity of Vector Functions (pg. 665)
% 		\begin{itemize}
% 		\item The function $\vect{r}(t)$ is \textbf{continuous at a point} $t_0$ if \[\lim_{t\to t_0}\vect{r}(t) = \vect{r}(t_0)\]
% 		\item
% 		The function $\vect{r}(t)$ is said to be \textbf{continuous on its domain} (or just \textbf{continuous}) if \[\lim_{t\to t_0}\vect{r}(t) = \vect{r}(t_0)\] for all $t_0$ in its domain.
% 		\item If $\vect{r}(t) = \<f(t),g(t),h(t)\>$, then $\vect{r}(t)$ is continuous if and only if $f(t),g(t),h(t)$ are all continuous.
% 		\end{itemize}
% 	\item Derivatives of Vector Functions (pg. 666)
% 		\begin{itemize}
% 		\item The \textbf{derivative} $\vect{r}'(t)$ of the vector function $\vect{r}(t)$ is given by the limit \[\vect{r}'(t) = \lim_{\Delta t \to 0} \frac{\vect{r}(t+\Delta t) - \vect{r}(t)}{\Delta t}\]
		
% 		This derivative is sometimes also written as $\displaystyle \frac{d\vect{r}}{dt}$.
% 		\item If $\vect{r}(t) = \<f(t),g(t),h(t)\>$, then it follows that  \[\vect{r}'(t) = \<f'(t),g'(t),h'(t)\>\]
% 		\item $\vect{r}(t)$ is \textbf{differentiable at a point} $t_0$ if $\vect{r}'(t_0)$ is well-defined.
% 		\item $\vect{r}(t)$ is \textbf{differentiable} if $\vect{r}'(t)$ is defined for every value of $t$ is in its domain.
% 		\item The curve given by $\vect{r}(t)$ is \textbf{smooth} if $\vect{r}(t)$ is differentiable, $\vect{r}'(t)$ is continuous, and $\vect{r}'(t) \not= 0$
% 		\item Given the vector $\vect{r}(t_0)$ pointing to a point on a curve, $\vect{r}'(t_0)$ is a \textbf{tangent vector} to the curve at that point.
% 		\item The \textbf{tangent line} to a curve given by the vector function $\vect{r}(t)$ when $t=t_0$ is given by the line passing through that point and parallel to the vector $\vect{r}'(t_0)$, with formula \[\vect{l}(t)=\vect{r}(t_0)+t\vect{r}'(t_0)\]
% 		\end{itemize}
% 	\newpage
% 	\item Vectors and Physics (pg. 667)
% 		\begin{itemize}
% 		\item If the vector function $\vect{r}(t)$ is the position function of a particle, then
% 			\begin{itemize}
% 			\item $\vect{v}(t) = \vect{r}'(t) = \frac{d\vect{r}}{dt}$ is the \textbf{velocity vector}
% 			\item $|\vect{v}(t)|$ is the \textbf{speed} of the particle
% 			\item $\frac{\vect{v}(t)}{|\vect{v}(t)|}$ is the \textbf{direction of motion of the particle}
% 			\item $\vect{a}(t) = \vect{v}'(t) = \vect{r}''(t)$ is the \textbf{acceleration vector}
% 			\end{itemize}
% 		\item Note that velocity is the product of its speed and its direction of motion, that is, \[\vect{v}(t) = |\vect{v}(t)|\left(\frac{\vect{v}(t)}{|\vect{v}(t)|}\right)\]
% 		\end{itemize}
	
% 	\item Differentiation Rules for Vector Functions (pg. 668)
	
% 		\begin{itemize}
% 		\item Constant Function Rule
% 		\[\frac{d}{dt} [\vect{C}] = \vect{0}\]
% 		\item Constant Multiple Rules
% 		\[\frac{d}{dt} [c\vect{u}(t)] = c\vect{u}'(t)\]
% 		\[\frac{d}{dt} [f(t)\vect{C}] = f'(t)\vect{C}\]
% 		\item Sum and Difference Rules
% 		\[\frac{d}{dt} [\vect{u}(t) \pm \vect{v}(t)] = \vect{u}'(t) \pm \vect{v}'(t)\]
% 		\item Scalar Product Rule
% 		\[\frac{d}{dt} [f(t)\vect{u}(t)] = f(t)\vect{u}'(t) + f'(t)\vect{u}(t)\]
% 		\item Dot Product Rule
% 		\[\frac{d}{dt} [\vect{u}(t) \cdot \vect{v}(t)] = \vect{u}(t)\cdot\vect{v}'(t) + \vect{u}'(t)\cdot\vect{v}(t)\]
% 		\item Cross Product Rule
% 		\[\frac{d}{dt} [\vect{u}(t) \times \vect{v}(t)] = \vect{u}(t)\times\vect{v}'(t) + \vect{u}'(t)\times\vect{v}(t)\]
% 		\item Chain Rule
% 		\[\frac{d}{dt} [\vect{u}(f(t))] =\vect{u}'(f(t))f'(t)\]
% 		\end{itemize}
		
% 	\item Note on the Chain Rule
	
% 	The Chain Rule above may be rewritten as: \[ \frac{d\vect{u}}{dt} = \frac{d}{dt} [\vect{u}(f(t))] = \vect{u}(f(t))f'(t) = \frac{d\vect{u}}{df}\frac{df}{dt} \]
	
% 	That is to say, given certain ``niceness'' properties on the function, vector (and scalar) derivatives ``cancel'' like fractions. Thus it is also true that for scalar variables $y,x$: \[\frac{1}{dy/dx} = \frac{dx}{dy}\]
	
% 	\item Derivative of a Constant Length Vector Function (pg. 669-70)
	
% 	If $\vect{r}(t)$ is a vector function of constant length, that is, there is a number $c$ where $|\vect{r}(t)|=c$ for all numbers $t$, then \[\vect{r}(t) \cdot \vect{r}'(t) = 0\]
	
% 	This is particularly true for unit vector functions (with constant length 1).
	
% 	\item \textbf{Suggested Exercises for 11.1}
% 		\begin{itemize}
% 		\item Position/Velocity/Acceleration Vectors: 1-14
% 		\end{itemize}
% 	\end{itemize}
	
% 	\newpage
	
% 	\centerline{\bf 11.2 Integrals of Vector Functions}
	
% 	Just as we can generalize derivatives to vectors, we can do the same with antiderivatives and integrals.
	
% 		\begin{itemize}
% 		\item Antiderivatives of Vector Functions (pg. 672)
% 			\begin{itemize}
% 			\item If $\vect{R}'(t)=\vect{r}(t)$, then $\vect{R}(t)$ is an \textbf{antiderivative} of $\vect{r}(t)$.
% 			\item The \textbf{indefinite integral} of $\vect{r}(t)$ is the collection of all the antiderivatives of $\vect{r}(t)$. It is written \[\int \vect{r}(t) \dvar{t}\]
% 			\item If $\vect{R}(t)$ is any antiderivative of $\vect{r}(t)$, then \[\int \vect{r}(t) \dvar{t} = \vect{R}(t) + \vect{C}\] where $\vect{C}=\<C_x,C_y,C_z\>$ represents an arbitrary constant vector of integration.
% 			\item It is easily proven that if $\vect{r}(t) = \<x(t),y(t),z(t)\>$, then \[\int \vect{r}(t) \dvar{t} = \<\int x(t) \dvar{t}, \int y(t) \dvar{t}, \int z(t) \dvar{t} \> \]
% 			\end{itemize}
% 		\item Definite Integrals (pg. 672-3)
% 			\begin{itemize}
% 			\item The \textbf{definite integral} of a vector function on an interval $[a,b]$ is defined to be \[\int^b_a \vect{r}(t) \dvar{t} = \<\int^b_a x(t) \dvar{t}, \int^b_a y(t) \dvar{t}, \int^b_a z(t) \dvar{t} \> \]
% 			\item The Fundamental Theorem of Calculus guarantees that \[\int^b_a \vect{r}(t)dt = \left[\vect{R}(t)\right]^b_a=\vect{R}(b)-\vect{R}(a)\] where $\vect{R}(t)$ is any antiderivative of $\vect{r}(t)$.
% 			\end{itemize}
			
% 		\item Initial Value Problems (pg. 673)
% 			\begin{itemize}
% 			\item If the derivative $\vect{r}'(t)$ of a function is known as well as some initial value $\vect{r}(t_0) = \vect{C_0}$, the original function $\vect{r}(t)$ may be found by finding the general antiderivative of $\vect{r}'(t)$ and solving for the constant of integration $\vect{C}$ by plugging in $t_0$ and setting the antiderivative equal to $\vect{C_0}$.
% 			\end{itemize}
		
% 		\item Ideal Projectile Motion (pg. 674-5)
% 			\begin{itemize}
% 			\item In ideal projectile motion, the unrealistic assumption is made that the only force acting on the projectile is a constant downward force of gravity $g$. (In real life other factors such as air resistance come into play.)
% 			\item Let the launch position of the projectile be the origin, the launch angle be $\alpha$, and the initial velocity vector of the launch be $\vect{v_0}$ (where $v_0$ is used to represent its magnitude $|\vect{v_0}|$).
% 			\item It follows that its initial position and velocity vectors (when $t=0$) are \[\vect{r_0} = \<0,0\> = \vect{0}\] \[\vect{v_0} = \<v_0\cos\alpha,v_0\sin\alpha\>\]
% 			\item The acceleration function for the projectile is \[\vect{a}(t) = \<0,-g\>\]
% 			\item The velocity function, found by integrating acceleration and using the initial value $\vect{v_0}$ at $t=0$, for the projectile is \[\vect{v}(t) = \<v_0\cos\alpha,-gt+v_0\sin\alpha\>\]
% 			\item The position function, found by integrating velocity and using the initial value $\vect{r_0}=\vect{0}$ at $t=0$, for the projectile is \[\vect{r}(t) = \<(v_0\cos\alpha)t,-\frac{1}{2}gt^2+(v_0\sin\alpha)t\>\]
% 			\item We may also express ideal projectile motion by the parametric equations \[x=(v_0\cos\alpha)t\] \[y=-\frac{1}{2}gt^2+(v_0\sin\alpha)t\]
			
% 			\item By substituting for $t$, we may find the following equation relating the projectile's $x$ and $y$ coordinates \[y = -\left(\frac{g}{2v_0^2\cos^2\alpha}\right)x^2+(\tan\alpha)x\] showing that its path is a parabola.
			
% 			\item By using applied optimization, we may find that the maximum height of the projectile is \[y_{max} = \frac{(v_0\sin\alpha)^2}{2g}\]
			
% 			\item By solving for $t$ when $y=0$, we may find that the flight time of the projectile is \[t_{tot} = \frac{2v_0\sin\alpha}{g}\]
% 			\item By solving for $x$ when $y=0$ or by plugging in $t_{tot}$, we may find that the range of the projectile is \[R = \frac{v_0^2}{g}\sin2\alpha\] (recall that $\sin2\alpha = 2\sin\alpha\cos\alpha$)
% 			\item If the initial position of the projectile is instead given by the vector $\vect{r_0}=\<x_0,y_0\>$, then the position function changes to \[\vect{r}(t)=\<(v_0\cos\alpha)t+x_0,-\frac{1}{2}gt^2+(v_0\sin\alpha)t+y_0\>\] (Note that the max height, etc., formulae only hold for when the initial position is the origin.)
% 			\end{itemize}
% 		\item \textbf{Suggested Exercises for 11.2}
% 			\begin{itemize}
% 			\item Vector function integrals: 1-6
% 			\item Vector function initial value problems: 7-12
% 			\item Ideal projectile motion: 15-21
% 			\end{itemize}
% 		\end{itemize}
	
% 	\newpage
	
% 	\centerline{\bf 11.3 Arc Length in Space}
	
% 	As we did with arcs in two dimensions (in Cal II), we may define the length of curved arcs in higher dimensions using integrals.
	
% 		\begin{itemize}
% 		\item Arc Length along a Space Curve (pg. 678-9)
% 			\begin{itemize}
% 			\item The \textbf{length} of a smooth curve $\vect{r}(t) = \<x(t),y(t),z(t)\>$ on the time interval $a \leq t \leq b$ may be approximated by \[L \approx \sum_{i=0}^n |\vect{r}(t_i+\Delta{t})-\vect{r}(t_i)|=\sum_{i=1}^n \left|\frac{\vect{r}(t_i+\Delta{t})-\vect{r}(t_i)}{\Delta{t}}\right|\Delta{t}\]
% 			\item Using the idea of a Riemann sum, we may then define \[L = \int_a^b \left|\lim_{\Delta{t}\to0}\frac{\vect{r}(t+\Delta{t})-\vect{r}(t)}{\Delta{t}}\right| \dvar{t} = \int_a^b |\vect{v}(t)| \dvar{t}\]
% 			\item If we choose a base time $t_0$, we may find a \textbf{directed distance} function (or \textbf{arc length parameter}) \[s(t) = \int_0^t |\vect{v}(\tau)|d\tau\] which gives the distance between the point at time $0$ to any point with time $t$.  If $t>0$ the directed distance is positive and if $t<0$ the directed distance is negative.
% 			\end{itemize}
% 		\item Speed on a Smooth Curve (pg. 680)
% 			\begin{itemize}
% 			\item By definition, if $\vect{r}(t)$ is smooth, then $\vect{v}(t)$ must be continuous.
% 			\item Thus by the Fundamental Theorem of Calculus, \[\frac{ds}{dt} = |\vect{v}(t)|\]
% 			\end{itemize}
% 		\item Unit Tangent Vector $\vect{T}$ (pg. 680-1)
% 			\begin{itemize}
% 			\item In 11.1 we learned that $\vect{r}'(t)$ (which equals $\vect{v}(t)$) gives a tangent vector to a curve.
% 			\item Thus the \textbf{unit tangent vector} $\vect{T}$ may be given by \[\vect{T} = \frac{\vect{v}}{|\vect{v}|}\]
% 			\item We may also think of $\vect{T}$ as the derivative of the position function with respect to the arclength parameter. Note \[\vect{T} = \frac{\vect{v}}{|\vect{v}|}=\frac{d\vect{r}/dt}{ds/dt} = \frac{d\vect{r}}{dt}\frac{dt}{ds} = \frac{d\vect{r}}{ds} \]
% 			\end{itemize}
% 		\item \textbf{Suggested Exercises for 11.3}
% 			\begin{itemize}
% 			\item Unit tangent vectors and arc length: 1-8
% 			\item Arc length parameter: 11-14
% 			\end{itemize}
% 		\end{itemize}
	
% 	\newpage
	
% 	\centerline{\bf 11.4 Curvature of a Curve}
	
% 	We wish to be able to find a quantity to describe how much a curve bends or turns.
	
% 	\begin{itemize}
% 	\item Curvature (pg. 683-4)
% 		\begin{itemize}
% 		\item We want curvature to represent how the curve is turning. This can be thought of as the rate of change of the unit tangent vector $\vect{T}$ with respect to arclength $s$.
% 		\item Thus, the \textbf{curvature} $\kappa$ (``kappa'') of a smooth curve is given by \[\kappa = \left|\frac{d\vect{T}}{ds}\right| \]
% 		\item Larger values of $\kappa$ correspond to sharper turns, while small values (close to $0$) of $\kappa$ designate a more straight path.
% 		\item It follows that \[\kappa = \frac{1}{|\vect{v}|}\left|\frac{d\vect{T}}{dt}\right|\]
% 		\end{itemize}
		
% 	\item Computational Formula (Section 11.5 pg. 692)
	
% 		\begin{itemize}
% 		\item In addition to the formulas for curvature $\kappa = \left|\frac{d\vect{T}}{ds}\right| = \frac{1}{|\vect{v}|}\left|\frac{d\vect{T}}{dt}\right|$, we may also  use $\vect{v}=\frac{ds}{dt}\vect{T}$ to prove that \[\kappa = \frac{|\vect{v} \times \vect{a}|}{|\vect{v}|^3}\]
% 		\end{itemize}
	
% 	\item Curvature of a Circle (pg. 684)
	
% 		\begin{itemize}
% 		\item The curvature of a circle with radius $a$, which may be parametrized \[\vect{r}(t)=\<a\cos t,a\sin t\>\] has curvature \[\kappa = \frac{1}{a}\] at every value of $t$.
% 		\end{itemize}
		
% 		\newpage
	
% 	\item Principal Unit Normal Vector (pg. 684-5)
% 		\begin{itemize}
% 		\item The \textbf{principal unit normal vector} $\vect{N}$ for a smooth curve in space is \[\vect{N} = \frac{d\vect{T}/ds}{|d\vect{T}/ds|}= \frac{1}{\kappa} \frac{d\vect{T}}{ds}\]
% 		\item This vector points in the direction of the turn of the curve.
% 		\item It is easily proven that \[\vect{N} = \frac{d\vect{T}/dt}{|d\vect{T}/dt|}\]
% 		\end{itemize}
	
% 	\item Circles of Curvature (pg. 685-6)
	
% 		\begin{itemize}
% 		\item The \textbf{circle of curvature} (sometimes called the \textbf{osculating circle}) of a plane curve given by a vector function $\vect{r}(t)$ at $t=t_0$ where $\kappa\not=0$ is the circle such that
% 			\begin{enumerate}
% 			\item The circle is tangent to the curve when $t=t_0$. (That is, they share the same tangent line at the point given by $t=t_0$.)
% 			\item The circle has the same curvature as the curve when $t=t_0$. (Thus, the circle's radius is the reciprocal of the curvature of the curve at $t=t_0$.)
% 			\item The circle lies on the concave (``inner'') side of the curve. (That is, the vector $\vect{N}$ given by $\vect{r}(t)$ at $t=t_0$ points toward the center of the circle.)
% 			\end{enumerate}
% 		\item Its radius $a$ is given by $\displaystyle\frac{1}{\kappa}$.
% 		\item Its center has the position vector $\vect{r}(t_0)+a\vect{N}$.
% 		\item Thus, if its center is written as $(x_0,y_0)$, it has the formula \[(x-x_0)^2+(y-y_0)^2=a^2\] and vector equation \[\vect{c}(t) = \<a\sin t+x_0,a\cos t+y_0\>\] for $0\leq t\leq 2\pi$.
% 		\end{itemize}
		
% 	\item \textbf{ Suggested Exercises for 11.4:}
	
% 		\begin{itemize}
% 		\item Find $\vect{T},\vect{N},\kappa$: 1-4, 9-16
% 		\item Circles of Curvature: 21-22
% 		\end{itemize}
		
% 	\end{itemize}
	
% 	\newpage
	
% 	\centerline{\bf 11.5 Tangental and Normal Components of Acceleration}
	
% 	\begin{itemize}
	
% 	\item The Binormal Vector (pg. 689)
	
% 		\begin{itemize}
% 		\item The \textbf{binormal vector} $\vect{B}$ to a curve in space is a unit vector perpendicular to the tangent and normal vectors.
% 		\item It is defined to be \[\vect{B} = \vect{T} \times \vect{N}\]
% 		\item The triple $\vect{T},\vect{N},\vect{B}$ provide a useful right-handed vector frame for calculating the paths of particles moving through space.
% 			\begin{itemize}
% 			\item $\vect{T}$ points forward
% 			\item $\vect{N}$ points left in the direction of the turn
% 			\item $\vect{B}$ points up
% 			\end{itemize}
% 		\end{itemize}
		
% 	\item Tangental and Normal Components of Acceleration (pg. 689-90)
	
% 		\begin{itemize}
% 		\item In section 11.3 we learned that \[\vect{v} = \left(\frac{ds}{dt}\right)\vect{T}\]
% 		\item By differentiating this equation with respect to $t$, we find \[\vect{a} = \left(\frac{d^2s}{dt^2}\right)\vect{T} + \kappa\left(\frac{ds}{dt}\right)^2\vect{N}+0\vect{B}\]
% 		\item Thus we may always express acceleration as a linear combination of the tangent and normal vectors to the curve. 
		
% 		Put another way, the acceleration vector lays in the plane determined by the vectors $\vect{T},\vect{N}$.
% 		\item If the acceleration vector is written as
% 		\[\vect{a} = a_T\vect{T}+a_N\vect{N}\]
% 		then its \textbf{tangential component} is
% 		\[a_T = \frac{d^2s}{dt^2} = \frac{d}{dt}|\vect{v}| \]
% 		and its \textbf{normal component} is
% 		\[a_N = \kappa\left(\frac{ds}{dt}\right)^2 = \kappa|\vect{v}|^2\]
% 		\item By using the Pythagorean Theorem, we find another formula for computing $a_N$ which doesn't require finding the curvature $\kappa$:
% 		\[a_N = \sqrt{|\vect{a}|^2 - a_T^2}\]
% 		\end{itemize}
	
% 	\item Torsion (pg. 691-2)
	
% 		\begin{itemize}
% 		\item We would like to define torsion so that it describes how much a curve is ``twisting''. That is, we should define it so that it describes the rate of change of the binormal vector $\vect{B}$ with respect to arclength $s$.
% 		\item The absolute value of torsion $|\tau|$ may be defined by \[|\tau| = \left|\frac{d\vect{B}}{ds}\right|\] However, we want to further define torsion to be positive for a twist to the right (away from $\vect{N}$), and negative otherwise. 
% 		\item By differentiation, we may find that \[\frac{d\vect{B}}{ds} = \vect{T} \times \frac{d\vect{N}}{ds}\]
% 		\item As $\frac{d\vect{B}}{ds}$ is orthogonal to $\vect{B}$ and $\vect{T}$, it must lie parallel to $\vect{N}$.  Therefore, we may write \[\frac{d\vect{B}}{ds} = \left(\pm\left|\frac{d\vect{B}}{ds}\right|\right)\vect{N}=(\pm|\tau|)\vect{N}\]
% 		\item Since we said we wanted a turn away from $\vect{N}$ to be positive, we use $-\tau$, giving the definition of \textbf{torsion} $\tau$ (``tau''): \[\frac{d\vect{B}}{ds} = (-\tau)\vect{N}\]
% 		\item Solving for $\tau$ yields \[\tau = -\frac{d\vect{B}}{ds}\cdot \vect{N}\]
% 		\item Replacing $ds$ with $\frac{ds}{dt}dt$ we find the formula \[\tau = -\frac{1}{|\vect{v}|}\left(\frac{d\vect{B}}{dt}\cdot\vect{N}\right)\]
% 		\end{itemize}
% 		\newpage
% 	\item Computational Formula (pg. 692)

% 		\begin{itemize}
% 		\item In advanced calculus, it is proved that torsion may also be computed by the formula
% 	\[
% 	\tau
% 	=
% 	\frac{
% 	\begin{array}{|ccc|}
% 	\dot{x} & \dot{y} & \dot{z} \\
% 	\ddot{x} & \ddot{y} & \ddot{z} \\
% 	\dddot{x} & \dddot{y} & \dddot{z}
% 	\end{array}
% 	}{
% 	|\vect{v}\times\vect{a}|^2
% 	}
% 	\]
% 	where $\dot{x}=\frac{dx}{dt}$.
% 		\end{itemize}
		
% 	\item \textbf{ Suggested Exercises for 11.5:}
	
% 		\begin{itemize}
% 		\item Finding tangental and normal components of acceleration: 1-6
% 		\item Finding $\vect{B}$ and $\tau$: 9-16
% 		\end{itemize}
	
% 	\end{itemize}
	
% 	\newpage
	
% 	\centerline{\bf 11.6 Velocity and Acceleration in Polar Coordinates}
	
% 	When describing (for instance) the motion of planets or other rotating objects, polar coordinates are often useful for modeling motion.
	
% 	Cylindrical coordinates extend the polar coordinate system into a 3D system.
	
% 	\begin{itemize}
	
% 	\item Polar Coordinates
	
% 		\begin{itemize}
% 		\item Recall that points in two dimensional space may be represented by the polar coordinate $(r,\theta)$, where $\theta$ represents the counter-clockwise angle from the $x$-axis and $r$ represents the directed distance from the origin.
% 		\item Unlike Cartesian space, polar coordinates are not unique. For instance, $(2,\frac{\pi}{4})$, $(-2,\frac{5\pi}{4})$, and $(2,\frac{9\pi}{4})$ all correspond to the same point.
% 		\end{itemize}
		
% 	\item Cylindrical Coordinates
	
% 		\begin{itemize}
% 		\item By adding a third parameter $z$, we get the cylindrical coordinate $(r,\theta,z)$.
% 		\item $r$ and $\theta$ correspond to the same position in the $xy$ plane as they do in polar coordinates, and $z$ corresponds to the distance from the $xy$ plane (as it does in 3D Cartesian coordinates).
% 		\end{itemize}
		
% 	\item Parametrization of a Polar/Cylindrical Curve
	
% 		\begin{itemize}
% 		\item Just as curves described with Cartesian coordinates may be parametrized, curves described with polar coordinates may also be parametrized.
% 		\item Motion on a 2D curve using polar coordinates may then be described using the functions $r(t),\theta(t)$.
% 		\item Motion on a 3D curve using cylindrical coordinates may then be described using the functions $r(t),\theta(t),z(t)$.
% 		\end{itemize}
		
% 	\item Motion in Polar and Cylindrical Coordinates (pg. 694)
	
% 		\begin{itemize}
% 		\item When a particle is positioned at the polar coordinate $(r,\theta)$ we may express its position, velocity, and acceleration by the unit vectors \[\vect{u}_r = \<\cos\theta,\sin\theta\> \text{ and } \vect{u}_\theta = \<-\sin\theta,\cos\theta\>\]
% 		\item If the particle's position is given by the vector $\vect{r}$, then it follows that \[\vect{r} = r\vect{u}_r\]
% 		\item By differentiating with respect to $\theta$, we find that \[\frac{d\vect{u}_r}{d\theta} = \<-\sin\theta,\cos\theta\> = \vect{u}_\theta \text{ and } \frac{d\vect{u}_\theta}{d\theta} = \<-\cos\theta,-\sin\theta\> = -\vect{u}_r\]
% 		\item If we differentiate with respect to $t$ to see how $\vect{u}_r$ and $\vect{u}_\theta$ change with respect to time, we get by the Chain Rule that \[\dot{\vect{u}}_r = \dot{\theta}\vect{u}_\theta \text{ and } \dot{\vect{u}}_\theta = -\dot\theta\vect{u}_r\] where $\dot x$ denotes $\frac{dx}{dt}$.
% 		\item This results in a formula for expressing $\vect{v}$ in terms of $\vect{u}_r,\vect{u}_\theta$: \[\vect{v} = \dot{r}\vect{u}_r + r\dot\theta\vect{u}_\theta\]
% 		\item Similarly, we may find \[\vect{a} = (\ddot{r} - r\dot\theta^2)\vect{u}_r + (r\ddot\theta + 2\dot{r}\dot\theta)\vect{u}_\theta\]
% 		\item For cylindrical coordinates, we note that \[\vect{r} = r\vect{u}_r + z\veck\] and thus \[\vect{v} = \dot{r}\vect{u}_r + r\dot\theta\vect{u}_\theta + \dot{z}\veck\] and \[\vect{a} = (\ddot{r} - r\dot\theta^2)\vect{u}_r + (r\ddot\theta + 2\dot{r}\dot\theta)\vect{u}_\theta+\ddot{z}\veck\]
% 		\item Like $\veci,\vecj,\veck$, the cylindrical unit vectors $\vect{u}_r,\vect{u}_\theta,\veck$ form a right hand frame such that
% 			\begin{itemize}
% 			\item $\vect{u}_r \times \vect{u}_\theta = \veck$
% 			\item $\vect{u}_\theta \times \veck = \vect{u}_r$
% 			\item $\veck \times \vect{u}_r = \vect{u}_\theta$
% 			\end{itemize}
% 		\end{itemize}
	
% 	\item \textbf{ Suggested Exercises for 11.6:}
	
% 		\begin{itemize}
% 		\item Expressing $\vect{v}$ and $\vect{a}$ in terms of $\vect{u}_r$ and $\vect{u}_\theta$: 1-5
% 		\end{itemize}
% 	\end{itemize}
	
% 	\newpage
	
% 	\centerline{\bf 11.6 Optional Material on Planetary Motion}
	
% 	\begin{itemize}
% 	\item Motion of Planets in Space (pg. 695-6)
	
% 		\begin{itemize}
% 		\item Newton's law of gravitation states that for two bodies, one of which is positioned at the origin and has mass $M$ (perhaps a sun), and another which is positioned at $\vect{r}$ and has mass $m$ (perhaps a planet), then the force $\vect{F}$ of gravitation between the two bodies is given by the equation \[\vect{F} = -\frac{GmM}{|\vect{r}|^2}\frac{\vect{r}}{|\vect{r}|}\] where $G$ is the universal gravitational constant $\approx 6.6726 \times 10^{-11}$ Nm$^2$kg$^{-2}$.
		
% 		\item Recalling Newton's second law ($\vect{F} = m\vect{a}$), we may then find \[\ddot{\vect{r}} = -\frac{GM}{|\vect{r}|^2}\frac{\vect{r}}{|\vect{r}|}\]
		
% 		\item Since $\ddot{\vect{r}}$ is a scalar multiple of $\vect{r}$, we have \[\vect{r} \times \ddot{\vect{r}} = \vect{0}\]
		
% 		\item And by the product rule, we find \[\frac{d}{dt}\left[\vect{r}\times\dot{\vect{r}}\right] = \vect{0}\] and thus \[\vect{r}\times\dot{\vect{r}} = \vect{C}\] for some constant vector $\vect{C}$.
		
% 		\item Since $\vect{r}$,$\dot{\vect{r}}$ lie in a plane perpendicular to $\vect{C}$, the planet must always move in a fixed plane including the center of its sun.
% 		\end{itemize}
	
% 	\item Kepler's First Law (pg. 696)
	
% 		\begin{itemize}
% 		\item Kepler's First Law states that the path of a revolving planet is an ellipse with the sun at one focus.
% 		\item The eccentricity of the ellipse is \[e = \frac{r_0v_0^2}{GM} - 1\] where $r_0$ is the minimum distance of the planet from the sun and $v_0$ is its linear speed at that moment.
% 		\item The polar equation of the ellipse is \[r = \frac{(1+e)r_0}{1+e\cos\theta}\]
% 		\end{itemize}
		
% 	\item Kepler's Second Law (pg. 696-7)
	
% 		\begin{itemize}
% 		\item Kepler's Second Law, known as the Equal Area Law, states that the planet always sweeps out a constant area inside the ellipse for equal amounts of time, regardless of position.
% 		\item This can be seen by observing that \[\frac{dA}{dt} = \frac{1}{2}r^2\dot\theta\] and \[r^2\dot\theta = r_0v_0\]
% 		\item As $\frac{dA}{dt}$ is constant with respect to time, the result follows.
% 		\end{itemize}
		
% 	\item Kepler's Third Law (pg. 697)
	
% 		\begin{itemize}
% 		\item Let $T$ represent the time it takes for a planet to orbit its sun and $a$ be the semimajor axis of its elliptical path. It then follows that \[\frac{T^2}{a^3}=\frac{4\pi^2}{GM}\]
% 		\end{itemize}
		
% 	\end{itemize}
	
% 	\newpage
	
% 	\centerline{\bf Sets}
	
% 	A few notes on the mathematical concept of a ``set''.
	
% 		\begin{itemize}
% 		\item A \textbf{set} is a collection of mathematical objects. Sets are often described as a list of objects surrounded by curly brackets.
% 		\item A set can contain any finite number of objects, or an infinite amount of objects.
% 			\begin{itemize}
% 			\item $\{1,f,\alpha\}$ is a finite set
% 			\item $\{4,6,8,10,\dots\}$ is an infinite set
% 			\end{itemize}
% 		\item The symbol $\in$ is used to describe membership in a set.
% 			\begin{itemize}
% 			\item $\alpha \in \{1,f,\alpha\}$
% 			\item $\alpha \not\in \{4,6,8,10,\dots\}$
% 			\end{itemize}
% 		\item The symbol $\subseteq$ is used to describe a \textbf{subset} of a set. If a set $A$ is a subset of a set $B$ (that is, $A\subseteq B$), then everything in $A$ must be in $B$.
% 			\begin{itemize}
% 			\item $\{1,2\} \subseteq \{1,2\}$
% 			\item $\{3,4,9\} \subseteq \{1,2,3,\dots\}$
% 			\item $\{2,3,-4\} \not\subseteq \{1,2,3,\dots\}$
% 			\item $\{1,3,5,\dots\} \subseteq \{1,2,3,\dots\}$
% 			\item $\{0,2,4,\dots\} \not\subseteq \{1,2,3,\dots\}$
% 			\end{itemize}
% 		\item Order and frequency do not matter in a set; only membership.
% 			\begin{itemize}
% 			\item $\{1,2,3\} = \{2,3,1\} = \{3,2,1\} = \{1,2,3,2\}$
% 			\end{itemize}
% 		\item If a set has no members, we say it is the \textbf{empty set} $\emptyset$.
% 		\item Sets are often defined by showing an ``example'' member and giving conditions for it following a colon.
% 			\begin{itemize} 
% 			\item $\mathbb{Z}  = \{n : n \text{ is an integer}\} = \{\dots,-2,-1,0,1,2,\dots\}$
% 			\item $\mathbb{P} = \{n : n \in \mathbb{Z} \text{ and } n > 0\} = \{1,2,3,\dots\}$
% 			\item $\mathbb{R} = \{x : x \text{ is a real number}\}$
% 			\item $[-2,4) = \{y \in \mathbb{R} : -2 \leq y < 4 \}$
% 			\end{itemize} \newpage
% 		\item A finite ordered list of objects $(x_1,x_2,\dots,x_n)$ is known as an \textbf{$n$-tuple}.
% 			\begin{itemize}
% 			\item The $2$-tuple $(3,-\frac{1}{2})$ is an ordered pair.
% 			\item The $3$-tuple $(\pi,1,-0.3)$ is an ordered triple.
% 			\end{itemize}
% 		\item The set $\mathbb{R}^n=\{(x_1,x_2,\dots,x_n) : x_i \text{ is a real number }\}$ represents the set of all $n$-tuples of real numbers.
% 			\begin{itemize}
% 			\item $(2,\pi,-3.1,0) \in \mathbb{R}^4$
% 			\item $\{(2,-1),(-e,7+\ln 2)\} \subseteq \mathbb{R}^2$
% 			\item $\mathbb{R}^2$ corresponds to the $xy$-plane, since every point in the plane corresponds to an ordered pair.
% 			\item Similarly, $\mathbb{R}^3$ corresponds to $xyz$-space.
% 			\end{itemize}
% 		\item A $\cap$ denotes the ``intersection'' of two sets, while a $\cup$ denotes the ``union'' of two sets.
% 			\begin{itemize}
% 			\item $A \cap B = \{x : x\in A \text{ and } x \in B\}$
% 			\item $(2,\infty) \cap [-2,4] = (2,4]$
% 			\item $A \cup B = \{x : x\in A \text{ or } x \in B\}$
% 			\item $(-1,1) \cup \mathbb{Z} = \{z : z \text{ is an integer or } -1<x<1 \}$
% 			\end{itemize}
% 		\item \textbf{Suggested Exercises}
% 			\begin{enumerate}
% 			\item Write ``all real numbers greater than 7'' in set notation and interval notation.
% 			\item Write ``all integers $n$ that make $n^2+3n$ nonnegative'' in set notation.
% 			\item Write ``all ordered pairs $(x,y)$ which are a solution to the inequality $2x+3y\geq 7$'' in set notation, and sketch this set in the plane.
% 			\item Write ``all ordered pairs such that the first coordinate is an integer and the second coordinate is any real number less than $-3$'' in set notation.
% 			\end{enumerate}
% 		\item \textbf{Further Reading}
% 			\begin{itemize}
% 			\item http://www.sosmath.com/algebra/inequalities/ineq02/ineq02.html
% 			\item http://en.wikipedia.org/wiki/Set-builder\_notation
% 			\end{itemize}
% 		\end{itemize}
	
	
% 	\newpage
	
% 	\centerline{\bf 12.1 Functions of Several Variables}
	
% 	So far, we've dealt with functions with a single input ($x$ or $t$, typically). We now generalize our concepts of limits, derivatives, and so on to functions which allow multiple inputs.
	
% 	\begin{itemize}
	
% 	\item Real-Valued Functions (pg. 702-3)
	
% 		\begin{itemize}
% 		\item Let $D\subseteq \mathbb{R}^n$ be any set of $n$-tuples. A \textbf{real-valued function} $f$ on with \textbf{domain} $D$ is a rule that assigns a real number \[w = f(x_1,x_2,\dots,x_n) \in \mathbb{R}\] to each $(x_1,x_2,\dots,x_n) \in D$.
% 		\item The set \[R = \{f(x_1,x_2,\dots,x_n) : (x_1,x_2,\dots,x_n) \in D\}\] is the \textbf{range} of the function. Note that this is the collection of every possible value given by $f$ for every value inside $D$.
% 		\item If our function accepts three or less inputs, we usually use the letters $x,y,z$ for $x_1,x_2,x_3$.
% 		\end{itemize}
		
% 	\item Domains and Ranges (pg. 703)
	
% 		\begin{itemize}
% 		\item The domain of a function is initially assumed to be some $\mathbb{R}^n$.
% 			\begin{itemize}
% 			\item $f(x,y) = x^2+y^3+4$ has domain $\mathbb{R}^2$.
% 			\item $f(x,y,z) = y^z + \cos(x)$ has domain $\mathbb{R}^3$.
% 			\end{itemize}
% 		\item If certain values of $\mathbb{R}^n$ result in division by zero, or a complex number, or otherwise don't make ``sense'', we assume the domain is restricted appropriately.
% 			\begin{itemize}
% 			\item $f(x,y) = \sqrt{x^2-y}$ has domain $\mathbb{R}^2$ except for when $x^2-y=0$. Thus, its domain is the set $\{(x,y):x^2-y\not=0\}$.
% 			\item $f(x,y,z) = \ln(x+y+z)$ has domain $\{(x,y,z) : x+y+z>0\}$.
% 			\end{itemize}
% 		\item The range of a function is given by all possible values that could be ``spit out'' by the function.
% 			\begin{itemize}
% 			\item $f(x,y) = \sqrt{x^2-y}$ has range $[0,\infty)=\{w: w\geq 0\}$.
% 			\item $f(x,y,z) = \ln(x+y+z)$ has range $\mathbb{R}$.
% 			\end{itemize}
% 		\end{itemize}
		
% 	\newpage
		
% 	\item Regions (pg. 703-4 and 707)
	
% 		\begin{itemize}
% 		\item A subset of the $xy$-plane ($\mathbb{R}^2$) is known as a \textbf{region}.
% 		\item Let $p\in\mathbb{R}^2$ and $\epsilon \in (0,\infty)$. A \textbf{ball} $B(p,\epsilon) \subseteq \mathbb{R}^2$ is the set of points \[B(p,\epsilon) = \{q \in \mathbb{R}^2 : \text{the distance between }p\text{ and }q\text{ is less than }\epsilon\}\] Its \textbf{center} is the point $p$ and its \textbf{radius} is $\epsilon$.
% 		\item A point $p\in\mathbb{R}^2$ is known as an \textbf{interior point} of a region $R$ if \textit{there exists some ball} containing $p$ that lies inside $R$.
% 		\item A point $p\in\mathbb{R}^2$ is known as a \textbf{boundary point} of a region $R$ if \textit{every ball} containing $p$ contains some points in $R$ and some points not in $R$.
% 		\item A point $p\in\mathbb{R}^2$ is known as an \textbf{exterior point} of a region $R$ if \textit{there exists some ball} containing $p$ that lies outside $R$.
% 			\begin{itemize}
% 			\item For every region $R$, every point is either an interior point, a boundary point, or an exterior point.
% 			\end{itemize}
% 		\item The \textbf{interior} of $R$ is the set \[\textrm{int}(R)=\{p : p \text{ is an interior point of } R\}\]
% 		\item The \textbf{boundary} of $R$ is the set \[\textrm{bd}(R)=\{p : p \text{ is a boundary point of } R\}\]
% 		\item The \textbf{exterior} of $R$ is the set \[\textrm{ext}(R)=\{p : p \text{ is an exterior point of } R\}\]
% 		\item A region $R$ is \textbf{open} if every point in $R$ is an interior point. Equivalently, $R$ is open if $\textrm{int}(R) = R$.
% 		\item A region $R$ is \textbf{closed} if every boundary point of $R$ is in $R$. Equivalently, $R$ is closed if $\textrm{bd}(R) \subseteq R$.
% 		\item A region $R$ is \textbf{bounded} if there exists some radius $\epsilon$ such that $R \subseteq B(O,\epsilon)$ where $O$ is the origin.
% 		\item A region $R$ is \textbf{unbounded} if it is not bounded.
% 		\item \textit{All of these concepts are also defined for $\mathbb{R}^3$.}
% 		\end{itemize}
% 	\newpage
% 	\item Level Curves, Graphs, and Contour Curves (pg. 705)
	
% 		\begin{itemize}
% 		\item Let $f(x,y)$ be a function.
% 			\begin{itemize}
% 			\item The set of points $\{(x,y):f(x,y)=c\}$ is called the \textbf{level $c$ curve} of $f$.
% 			\item The set of points $\{(x,y,f(x,y)): (x,y)\in \textrm{Dom}(f)\}$ is called the \textbf{graph} of $f$. It is also known as the \textbf{surface} $z=f(x,y)$.
% 			\item The set of points $\{(x,y,c): f(x,y)=c\}$ is the \textbf{contour $c$ curve} of $f$.
% 			\end{itemize}
% 		\item Note that a contour curve lays on the graph of $f$, and the corresponding level curve is its ``shadow'' on the $xy$-plane.
% 		\end{itemize}
	
% 	\item Level Surfaces (pg. 705-6)
	
% 		\begin{itemize}
% 		\item Let $f(x,y,z)$ be a function. The set of points $\{(x,y,z):f(x,y,z)=c\}$ is called the \textbf{level $c$ surface} of $f$.
% 		\end{itemize}
	
% 	\item \textbf{ Suggested Exercises for 12.1:}
	
% 		\begin{itemize}
% 		\item Identifying and describing domains, ranges, level curves, boundaries: 1-12
% 		\item Relating level curves to graphs: 13-18
% 		\item Sketching surfaces and level curves: 19-28
% 		\item Finding level curves through a point: 29-32
% 		\item Sketching level surfaces: 33-40
% 		\item Finding level surfaces through a point: 41-44
% 		\end{itemize}
		
% 	\end{itemize}
	
% 	\newpage
	
% 	\centerline{\bf 12.2 Limits and Continuity in Higher Dimensions}
	
% 	The idea of a limit for a function $f(x)$ can also be extended to higher-dimensional functions $f(x,y)$ and $f(x,y,z)$.
	
% 	\begin{itemize}
	
% 	\item Limits (pg. 711-2)
% 		\begin{itemize}
% 		\item If the value of the vector function $f(P)$ becomes arbitrarily close to the number $L$ as points $P$ close to $P_0$ are plugged into the function, then the \textbf{limit of $f(P)$ as $P$ approaches $P_0$} is $L$, or \[\lim_{P\to P_0} f(P) = L\]
% 			\begin{itemize}
% 			\item Precise definition: \newline
% 			Suppose there exists a function $\delta(\epsilon)$ so that for all positive numbers $\epsilon>0$ and all numbers $P$ where $|f(P)-L|<\epsilon$, it follows that $|\vect{P}-\vect{P_0}|<\delta(\epsilon)$. Then we say that $\lim_{P\to P_0} f(P) = L$.
% 			\end{itemize}
% 		\item If the limit is to be well-defined, the values of $f$ must approach $L$ regardless as to which direction we approach $p_0$.
% 		\end{itemize}
		
% 	\item Limit Laws (pg. 712)
	
% 			\begin{enumerate}
% 			\item Sum/Difference Law \[\lim_{p\to p_0}(f(p)\pm g(p)) = \lim_{p\to p_0}f(p) \pm \lim_{p\to p_0}g(p)\]
% 			\item Product Law \[\lim_{p\to p_0}(f(p)\cdot g(p)) = \lim_{p\to p_0}f(p) \cdot \lim_{p\to p_0}g(p)\]
% 			\item Constant Multiple Law \[\lim_{p\to p_0}(kf(p)) = k\lim_{p\to p_0}f(p)\]
% 			\item Quotient Law \[\lim_{p\to p_0}\frac{f(p)}{g(p)} = \frac{\ds \lim_{p\to p_0}f(p)}{\ds \lim_{p\to p_0}g(p)}\]
% 			\item Power Law (for $r,s\in \mathbb{Z}$) \[\lim_{p\to p_0}(f(p))^{r/s} = \left(\lim_{p\to p_0}f(p)\right)^{r/s}\]
% 			\end{enumerate}
					
% 	\item Computing Limits (pg. 712-4)
		
% 		\begin{itemize}
		
% 		\item Using the Limit Laws to ``split up'' the variables, when able, allow us to compute many limits as we did in Calculus I.
		
% 			\begin{itemize}
% 			\item $\ds \lim_{(x,y)\to(x_0,y_0)} f(x) = \lim_{x\to x_0} f(x)$
% 			\item $\ds \lim_{(x,y,z)\to(x_0,y_0,z_0)} f(y) = \lim_{y\to y_0} f(y)$
% 			\end{itemize}
		
% 		\item Reducing multivariable limits to limits of a single variable allow us to use tricks such as L'Hopital's Rule. 
			
% 		\item Factoring and canceling tricks that we used for limits in Calculus I also work for limits of higher dimensions.
			
% 		\end{itemize}
	
% 	\item Limits Along Curves (pg. 715-6)
	
% 		\begin{itemize}
% 		\item It's important to note that a multi-dimensional limit must hold regardless of the path of approach.
% 		\item If $\ds\lim_{p\to p_0} f(p)$ exists, then it must exist along any curve containing the point $p$.
% 		\item If $\ds\lim_{p\to p_0} f(p)$ is found to attain two different values along two different curves, then the limit cannot exist.
% 		\end{itemize}
	
% 	\item Continuity (pg. 714-5, 716)
	
% 		\begin{itemize}
% 		\item A function $f(p)$ is \textbf{continuous at the point} $p_0$ if $\ds \lim_{p\to p_0}f(p) = f(p_0)$. 
% 		\item A function is \textbf{continuous} if it is continuous at every point of its domain.
% 		\item If $f(p)$ is a continuous real-valued function of any number of variables, and $g(x)$ is a continuous real-valued function of a single variable, then its composition $h(p) = g(f(p))$ is a continuous function.
% 		\end{itemize}
	
% 	\item \textbf{ Suggested Exercises for 12.2:}
	
% 		\begin{itemize}
% 		\item Computing limits: 1-26
% 		\item Checking continuity: 27-34
% 		\item Showing limits don't exist: 35-42
% 		\end{itemize}
	
% 	\end{itemize}
	
% 	\newpage
	
% 	\centerline{\bf 12.3 Partial Derivatives}
	
% 	For one-dimensional functions, the derivative was considered the rate of change at a point. However, for multi-dimensional functions, the rate of change depends on the path of approach. In this section, we consider the rate of change as we only vary a single variable.
	
% 	\begin{itemize}
	
% 	\item Partial Derivatives (pg. 719-21)
	
% 		\begin{itemize}
% 		\item Let $f(x_1,\dots,x_n)$ be a real-valued function.  The \textbf{partial derivative of $f$ with respect to $x_i$} is the limit \[\frac{\partial f}{\partial x_i}=f_{x_i}=\lim_{h\to 0}\frac{f(x_1,\dots,x_i+h,\dots,x_n)-f(x_1,\dots,x_i,\dots,x_n)}{h}\] provided the limit exists.
% 		\item For example, the partial derivative of $f(x,y,z)$ with respect to $y$ is the limit \[\frac{\partial f}{\partial y} = f_y = \lim_{h\to 0}\frac{f(x,y+h,z)-f(x,y,z)}{h}\]
% 		\item Suppose we hold $x,z$ constant in $f(x,y,z)$. Then $f(x,y,z)=F(y)$ and \[f_y = \lim_{h\to 0}\frac{f(x,y+h,z)-f(x,y,z)}{h}=\lim_{h\to 0}\frac{F(y+h)-F(y)}{h}=F'(y)\]
% 		\item So, \textbf{to compute partial derivatives with respect to a variable, treat all other variables as constants and differentiate as normal}.
% 		\item To find the rate of change of $f(x,y,z)$ at the point $(x_0,y_0,z_0)$ as we vary only $y$, we may compute \[\left.\frac{\partial f}{\partial y} \right|_{(x_0,y_0,z_0)}=f_y(x_0,y_0,z_0)=\lim_{h\to 0}\frac{f(x_0,y_0+h,z_0)-f(x_0,y_0,z_0)}{h}\]
% 		\end{itemize}
		
% 	\item Higher Order Partial Derivatives (pg. 725-7)
	
% 		\begin{itemize}
% 		\item A partial derivative is a new function which we may wish to differentiate partially again to see how it changes as we vary a variable.
% 		\item The resulting notation is \[\frac{\partial^2 f}{\partial x\partial y} = \frac{\partial}{\partial x}\left[ \frac{\partial f}{\partial y} \right] = (f_y)_x = f_{yx}\] \[\frac{\partial^2 g}{\partial z\partial z}= \frac{\partial^2 g}{\partial z^2}= g_{zz}\]
% 		\item The \textbf{Mixed Derivative Theorem} says that if $f,f_x,f_y,f_{xy},f_{yx}$ are all defined and continuous for every point in an open region containing the point $(a,b)$, then \[f_{xy}(a,b)=f_{yx}(a,b)\]
% 		\item Thus, if $f,f_x,f_y,f_{xy},f_{yx}$ are defined and continuous on the domain of $f_{xy}$ and $f_{yx}$, then \[f_{xy}=f_{yx}\]
% 		\end{itemize}
		
% 	\item Differentiability (pg. 757-8)
	
% 		\begin{itemize}
% 		\item The function
% 	\[
% 	f(x,y) = \left\{
% 	\begin{array}{l@{\ :\ }l}
% 	1 & x=0 \text{ or } y=0 \\
% 	0 & x\not=0 \text{ and } y\not=0
% 	\end{array}
% 	\right.
% 	\]
% 		has well-defined partial derivatives $f_x,f_y$ at $(0,0)$, but is not itself continuous at $(0,0)$.
% 		\item We want differentiability to imply continuity as it does for one-dimensional functions. So, we need a stronger requirement for differentiability.
% 		\item We say $f$ is \textbf{strongly differentiable at a point $p$} if there is an open region containing $p$ such that $f_x,f_y$ are defined and continuous in that region.
% 		\item Then $f$ is \textbf{strongly differentiable} if it is strongly differentiable at every point in its domain.
% 		\item It follows from this definition that if $f$ is strongly differentiable at a point, then it is continuous there.
% 		\end{itemize}
	
% 	\item \textbf{Suggested Exercises for 12.3:}
	
% 		\begin{itemize}
% 		\item Finding first-order partial derivatives: 1-38
% 		\item Finding second-order partial derivatives: 41-50
% 		\item Finding partial derivatives from the limit definition: 53-56
% 		\end{itemize}
	
% 	\end{itemize}
	
% 	\newpage
	
% 	\centerline{\bf 12.4 The Chain Rule}
	
% 	For scalar functions, the chain rule \[\frac{df}{dx}=\frac{d}{dx}[f(u)]=f'(u)u'(x)=\frac{df}{du}\frac{du}{dx}\] describes how to differentiate a composition of functions.  We will now address the differentiation of composition of functions which may have multiple inputs.
	
% 	\begin{itemize}
	
% 	\item Gradient Vector Function (pg. 742 [Section 12.5])
	
% 		\begin{itemize}
% 		\item The \textbf{gradient vector function} (or simply \textbf{gradient}) of a real-valued function $f(x_1,\dots,x_n)$ is the $n$-dimensional vector function \[\nabla f = \left\langle f_{x_1}, \dots, f_{x_n} \right\rangle\]
% 		\item For example, the gradient of $f(x,y,z)$ is \[\nabla f = \left\langle f_x,f_y,f_z \right\rangle = \left\langle \frac{\partial f}{\partial x}, \frac{\partial f}{\partial y}, \frac{\partial f}{\partial z} \right\rangle\]
% 		\end{itemize}
	
% 	\item Chain Rule
	
% 		\begin{itemize}
% 		\item If $f$ is a function of one or more variables given by $\vect{r}=\<x_1,x_2,\dots\>$, each of which is a function of the single variable $t$, then \[\frac{df}{dt}=\nabla{f}\cdot\frac{d\vect{r}}{dt}=\frac{\partial f}{\partial x_1}\frac{dx_1}{dt}+\frac{\partial f}{\partial x_2}\frac{dx_2}{dt}+\dots\]
% 		\item If $f$ is a function of one or more variables given by $\vect{r}=\<x_1,x_2,\dots\>$, each of which is a function of the variables $t_1,t_2,\dots$, then \[\frac{\partial f}{\partial t_i}=\nabla{f}\cdot\frac{\partial\vect{r}}{\partial t_i}=\frac{\partial f}{\partial x_1}\frac{\partial x_1}{\partial t_i}+\frac{\partial f}{\partial x_2}\frac{\partial x_2}{\partial t_i}+\dots\]
% 		\end{itemize}
		
% 	\item Differentiation by Substitution
	
% 		\begin{itemize}
% 		\item The above Chain Rules are useful in some cases, but are not always necessary.
% 		\item For example, if $f(x,y,z) = x^2y+y\ln z$, $x(t) = t^3$, $y(t) = t^2$, and $z(t) = t+1$, you may find it easier to differentiate \[f(x(t),y(t),z(t)) = (t^3)^2(t^2)+(t^2)\ln(t+1) = t^8 + t^2\ln(t+1)\] with respect to $t$ rather than applying the chain rule.
% 		\end{itemize}
		
% 	\newpage
		
% 	\item Implicit Differentiation (pg. 735-6)
	
% 		\begin{itemize}
% 		\item Suppose that $f(x,y)$ is differentiable and $f(x,y)=c$ defines $y$ as a function of $x$. It follows that \[\frac{\partial f}{\partial x}=f_x\frac{dx}{dx}+f_y\frac{dy}{dx} = 0\]
% 		\item Thus it follows that $\ds\frac{dy}{dx} = -\frac{f_x}{f_y}$.
% 		\end{itemize}
		
% 	\item \textbf{Suggested Exercises for 12.4:}
	
% 		\begin{itemize}
% 		\item Finding $\frac{dw}{dt}$ for $w=f(x(t),y(t),z(t))$: 1-6
% 		\item Finding partial derivatives for compositions of multi-variable functions: 7-12, 33-38
% 		\item Using partial derivatives for implicit differentiation: 25-28
% 		\end{itemize}
	
% 	\end{itemize}
	
% 	\newpage
	
% 	\centerline{\bf 12.5 Directional Derivatives and Gradient Vectors}
	
% 	Using a partial derivative such as $\frac{\partial f}{\partial y}$ we may describe the rate of change of the function $f$ as we vary that individual variable. We now wish to be able to describe the rate of change of a variable as we vary the input values along any curve in the domain.
	
% 	\begin{itemize}
	
% 	\item Directional Derivative
	
% 		\begin{itemize}
% 		\item We wish to define the directional derivative to be the rate of change of $f$ at a point as we move through a line passing through that point.
% 		\item Let $\vect{u}$ be a unit vector and $f$ be a function of several variables. Note that the line passing through a point $P_0$ in the direction of $\vect{u}$ has equation $\vect{r}(s)=\vect{P_0}+s\vect{u}$.
% 		\item Thus the rate of change of $f$ with respect to $s$ along this line is \[\frac{df}{ds}=\nabla f \cdot \frac{d\vect{r}}{ds} = \nabla f \cdot \vect{u}\]
% 		\item We call this the \textbf{directional derivative of $f$ in the direction of $\vect{u}$} and write it \[ \left(\frac{df}{ds}\right)_{\vect{u}} = D_{\vect{u}}f = \nabla f \cdot \vect{u}\]
% 		\item To find the directional derivative at a point $P_0$, evaluate the gradient at that point. \[ \left(\frac{df}{ds}\right)_{\vect{u},P_0} = \nabla f_{P_0} \cdot \vect{u}\]
% 		\end{itemize}
		
% 	\item Directional Derivative Properties (pg. 743-4)
	
% 		\begin{itemize}
% 		\item We note that we may rewrite the directional derivative as \[\left(\frac{df}{ds}\right)_{\vect{u}}=\nabla f \cdot \vect{u}=|\nabla f||\vect{u}|\cos \theta = |\nabla f|\cos \theta\] where $\theta$ is the angle between $\nabla f$ and $\vect{u}$.
% 		\item Thus the angle between $\nabla f$ and $\vect{u}$ determines the magnitude of the rate of change in the direction of $\vect{u}$.
% 		\item For a fixed point $p_0$, there is a bounded range of possible values for directional derivatives. That is:
% 			\begin{enumerate}
% 			\item When the angle between $\nabla f$ and $\vect{u}$ is $0$ radians, $\left(\frac{df}{ds}\right)_{\vect{u},p_0}$ is maximized at $|\nabla f_{p_0}|$.
% 			\item When the angle between $\nabla f$ and $\vect{u}$ is $\frac{\pi}{2}$ radians, $\left(\frac{df}{ds}\right)_{\vect{u},p_0}$ vanishes to $0$.
% 			\item When the angle between $\nabla f$ and $\vect{u}$ is $\pi$ radians, $\left(\frac{df}{ds}\right)_{\vect{u},p_0}$ is minimized at $|\nabla f_{p_0}|$.
% 			\end{enumerate}
% 		\end{itemize}
		
% 	\item Gradients are Normal Vectors to Level Curves (pg. 744)
% 		\begin{itemize}
% 		\item Consider the level-$c$ curve of $f(x,y)$ given by the equation \[f(x,y)=c\] and suppose it can be described by the vector function \[\vect{r}(t) = \langle x(t),y(t) \rangle\]
% 		\item Then it follows that for all $t$, \[f(x(t),y(t))=c\]
% 		\item Differentiating with respect to $t$, we find
% 	\[
% 	\begin{array}{r@{\ =\ }l}\ds
% 	\frac{d}{dt}[f(x,y)]
% 	&\ds
% 	\frac{d}{dt}[c]
% 	\\\ds
% 	\nabla f \cdot \frac{d\vect{r}}{dt}
% 	&\ds
% 	0
% 	\end{array}
% 	\]
% 		\item So $\nabla f$ is orthogonal to a tangent vector of the local level curve, so it is normal to the level curve itself.
% 		\end{itemize}
	
% 	\item Find Tangent Vectors and Lines to Level Curves (pg. 744-5)
% 		\begin{itemize}
	
% 		\item In section 12.4 we learned the level curve $f(x,y)=c$ has slopes given by $-\frac{f_x}{f_y}$.
% 		\item Thus the tangent line to the level curve of $f$ at $(x_0,y_0)$ is given by \[y-y_0=m(x-x_0)\] where $m = -\frac{f_x(x_0,y_0)}{f_y(x_0,y_0)}$.
% 		\end{itemize}
% 	\newpage
% 	\item Gradient Rules (pg. 745)
	
% 			\begin{enumerate}
% 			\item Constant Multiple Rule
% 			\[\nabla(kf)=k\nabla f\]
% 			\item Sum Rule
% 			\[\nabla(f+g)=\nabla f+\nabla g\]
% 			\item Difference Rule
% 			\[\nabla(f-g)=\nabla f-\nabla g\]
% 			\item Product Rule
% 			\[\nabla(fg)=g(\nabla f)+f(\nabla g)\]
% 			\item Quotient Rule
% 			\[\nabla\left(\frac{f}{g}\right)=\frac{g(\nabla f)-f(\nabla g)}{g^2}\]
% 			\end{enumerate}
	
% 	\item \textbf{Suggested Exercises for 12.5:}
	
% 		\begin{itemize}
% 		\item Finding $\nabla f$ at a point: 1-8
% 		\item Finding directional derivatives: 9-16
% 		\item Finding the direction of maximal/minimal rate of change: 17-22
% 		\item Finding the tangent line to a level curve: 23-26
% 		\item Finding the direction of no instantaneous change: 27-28
% 		\end{itemize}
	
% 	\end{itemize}
	
% 	\newpage
	
% 	\centerline{\bf 12.6 Tangent Planes and Differentials}
	
% 	Much like we can define the tangent line to a curve in two dimensions, we may define a tangent plane to a surface in three dimensions.
	
% 	(We will omit the portions of this section dealing with differentials.)
		
% 	\begin{itemize}
	
% 	\item Normal Vector to a Level Surface $f(x,y,z)=c$ (pg. 747-8)
	
% 		\begin{itemize}
% 		\item As in the previous section, we may find that if we choose any curve $\vect{r}(t)=\<x(t),y(t),z(t)\>$ on the level-$c$ surface $f(x,y,z)=c$, then we may find 
% 	\[
% 	\begin{array}{r@{\ =\ }l}\ds
% 	\frac{d}{dt}[f(x,y,z)]
% 	&\ds
% 	\frac{d}{dt}[c]
% 	\\\ds
% 	\nabla f \cdot \frac{d\vect{r}}{dt}
% 	&\ds
% 	0
% 	\end{array}
% 	\]
% 		\item So we find  that $\nabla f$ is normal to every vector which is tangent to the level surface, and is thus normal to the level surface.
% 		\end{itemize}
		
% 	\item Normal Vector to the Surface $z=f(x,y)$ (pg. 749)
	
% 		\begin{itemize}
% 		\item To find the normal vector to $z=f(x,y)$, we note that we may rewrite this as $f(x,y)-z=0$.
% 		\item So if we let $g(x,y,z)=f(x,y)-z$, then we may find the normal vector to the level-$0$ surface $g(x,y,z)=f(x,y)-z=0$, which is given by \[\nabla g=\<g_x,g_y,g_z\>=\<f_x,f_y,-1\>\]
% 		\end{itemize}
		
% 	\item List of Normal Vectors, Normal Lines and Tangent Planes to Surfaces (pg. 748-9)
	
% 		\begin{itemize}
% 		\item Recalling the formulas from section 10.5, we find the following equations:
% 			\begin{itemize}
% 			\item Normal Vector to $f(x,y,z)=c$ at $P_0=(x_0,y_0,z_0)$
% 			\[\nabla f_{P_0}=\<f_x(P_0),f_y(P_0),f_z(P_0)\>\]
% 			\item Normal Line to $f(x,y,z)=c$ at $P_0=(x_0,y_0,z_0)$
% 			\[\vect{r}(t) = \vect{P_0} + t\nabla f_{P_0}\]
% 			\item Tangent Plane to $f(x,y,z)=c$ at $P_0=(x_0,y_0,z_0)$
% 			\[f_x(P_0)(x-x_0)+f_y(P_0)(y-y_0)+f_z(P_0)(z-z_0)=0\]
			
% 			\item Normal Vector to $z=f(x,y)$ at $P_0=(x_0,y_0,z_0)$
% 			\[\<f_x(P_0),f_y(P_0),-1\>\]
% 			\item Normal Line to $z=f(x,y)$ at $P_0=(x_0,y_0,z_0)$
% 			\[\vect{r}(t) = \vect{P_0} + t\<f_x(P_0),f_y(P_0),-1\>\]
% 			\item Tangent Plane to $z=f(x,y)$ at $P_0=(x_0,y_0,z_0)$
% 			\[f_x(P_0)(x-x_0)+f_y(P_0)(y-y_0)-(z-z_0)=0\]
% 			\end{itemize}
% 		\end{itemize}
		
% 	\item Tangent Line to Curve of Intersection of Two Surfaces (pg. 749-50)
	
% 		\begin{itemize}
% 		\item The line tangent to the the curve of intersection of two surfaces $f(x,y,z)=c$ and $g(x,y,z)=d$ at a point $P_0=(x_0,y_0,z_0)$ must lay on the tangent planes to both surfaces.
% 		\item Thus it must be the line of intersection given by the tangent planes, which have normal vectors $\nabla f_{P_0},\nabla g_{P_0}$.
% 		\item Recalling Section 10.5, a vector parallel to the line of intersection of planes with normal vectors $\nabla f_{P_0},\nabla g_{P_0}$ is given by the cross product \[\nabla f_{P_0}\times\nabla g_{P_0}\]
% 		\item The vector equation for the tangent line to the curve of intersection of two surfaces $f(x,y,z)=c$ and $g(x,y,z)=d$ at a point $P_0=(x_0,y_0,z_0)$ is then \[\vect{r}(t) = \vect{r}_0 + t(\nabla f_{P_0}\times\nabla g_{P_0})\]
% 		\end{itemize}
		
% 	\item \textbf{Suggested Exercises for 12.6:}
	
% 		\begin{itemize}
% 		\item Finding tangent planes \& normal lines to surfaces of the form $f(x,y,z)=c$: 1-8
% 		\item Finding tangent planes \& normal lines to surfaces of the form $z=f(x,y)$: 9-12
% 		\item Finding tangent lines to curves of intersection: 13-18
% 		\end{itemize}
		
% 	\end{itemize}
	
% 	\newpage
	
% 	\centerline{\bf 12.7 Extreme Values and Saddle Points}
	
% 	To find the local extrema of functions of a single variable, we looked at the critical points given by the zeros of the derivative. In this section, we find that the partial derivatives give us places to look for the extrema of functions of multiple variables.
	
% 	\begin{itemize}
	
% 	\item Local Extreme Values (pg. 756-7)
	
% 		\begin{itemize}
% 		\item Let $f(p)$ be a function of many variables defined on a region containing the point $p_0$.
% 			\begin{itemize}
% 			\item If $f(p_0)\geq f(p)$ for all points $p$ in an open region $R$ containing $p_0$, then the number $f(p_0)$ is called a \textbf{local maximum value} at $p_0$ of $f$. (That is, the value of $f$ is a local maximum if it is the largest nearby value.)
% 			\item If $f(p_0)\leq f(p)$ for all points $p$ in an open region $R$ containing $p_0$, then the number $f(p_0)$ is called a \textbf{local minimum value} at $p_0$ of $f$. (That is, the value of $f$ is a local minimum if it is the smallest nearby value.)
% 			\end{itemize}
% 		\item Local max/mins are also known as \textbf{local extrema}.
% 		\end{itemize}
		
% 	\item First Derivative Test for Local Extreme Values (pg. 757)
	
% 		\begin{itemize}
% 		\item Suppose $f(x,y)$ is a function of two variables, $(a,b)$ is in the interior of the domain of $f$, $f(a,b)$ is a local extreme value of $f$, and $f_x(a,b),f_y(a,b)$ exist.
% 		\item It follows that $g(x)=f(x,b)$ has a local extremum of $g(a)$. Note that $g'(x)=f_x(x,y)$.
% 		\item And by the first derivative test for single-variable functions, we know that $g'(a)=f_x(a,b)=0$ or the derivative does not exist.
% 		\item Since this could be extended for any number of variables, and is true for every variable, we have the following theorem:
% 			\begin{itemize}
% 			\item The \textbf{First Derivative Test for Local Extreme Values} says, for $p_0$ in the interior of the domain of $f$, that $f(p_0)$ may only be a local extreme value of $f(x_1,\dots,x_n)$ in the case that either $f_{x_i}(p_0)$ does not exist for \textit{at least one} variable $x_i$ of the function, or $f_{x_i}(p_0)=0$ for \textit{every} variable $x_i$ of the function.
% 			\end{itemize} 
% 		\end{itemize}
% 	\newpage
% 	\item Horizontal Tangent Planes (pg. 757-8)
	
% 		\begin{itemize}
% 		\item Recall that $z=f(x,y)$ has the tangent plane \[f_x(a,b)(x-a)+f_y(a,b)(y-b)-(z-f(a,b))=0\] at the point $(a,b)$. If $f(a,b)$ is a local extreme value, then we may use the first derivative test to rewrite this equation as \[0(x-a)+0(y-b)-(z-f(a,b))=0\] \[\Rightarrow z=f(a,b)\]
% 		\item This shows that if $f(a,b)$ is a local extreme value, then $f(x,y)$ has a horizontal tangent plane at the point $(a,b,f(a,b))$.
% 		\end{itemize}
		
% 	\item Critical Points (pg. 758)
	
% 		\begin{itemize}
% 		\item If $f(x_1,\dots,x_n)$ is a function of many variables, and either $f_{x_i}(p_0)$ does not exist for \textit{at least one} variable $x_i$ of the function, or $f_{x_i}(p_0)=0$ for \textit{every} variable $x_i$ of the function, then we say $p_0$ is a \textbf{critical point} of the function $f$.
% 		\item We may then rewrite the first derivative test thusly:
% 			\begin{itemize}
% 			\item The \textbf{First Derivative Test for Local Extreme Values} says, for $p_0$ in the interior of the domain of $f$, that $f(p_0)$ may only be a local extreme value of $f(x_1,\dots,x_n)$ in the case that $p_0$ is a critical point of $f$.
% 			\end{itemize}
% 		\end{itemize}
		
% 	\item Saddle Points (pg. 758)
	
% 		\begin{itemize}
% 		\item Not every critical point yields a local extreme value.
% 		\item If $(a,b)$ is a critical point of a function $f(x,y)$ of two variables, but $f(a,b)$ is not an extreme value of $f$, then we say $(a,b,f(a,b))$ is a \textbf{saddle point} at $(a,b)$ of $f$.
% 		\end{itemize}
		
% 	\item Second Derivative Test for Local Extreme Values (pg. 759)
	
% 		\begin{itemize}
% 		\item Let $f(x,y)$ be a function of two variables. The function \[\textrm{Dis}(x,y)=f_{xx}(x,y)f_{yy}(x,y)-f_{xy}^2(x,y)=\begin{array}{|cc|}f_{xx}(x,y)&f_{xy}(x,y)\\f_{xy}(x,y)&f_{yy}(x,y)\end{array}\] is called the \textbf{discriminant} or \textbf{Hessian} of $f$.
% 		\item The \textbf{Second Derivative Test for Local Extreme Values} says that if $f(x,y)$ is a function of two variables, $f$ and its first and second partial derivatives are all defined and continuous in an open region about $(a,b)$, and $f_x(a,b)=f_y(a,b)=0$. Then:
% 			\begin{enumerate}
% 			\item If $\textrm{Dis}(a,b)>0$ and $f_{xx}(a,b)<0$, then $f(a,b)$ is a local maximum.
% 			\item If $\textrm{Dis}(a,b)>0$ and $f_{xx}(a,b)>0$, then $f(a,b)$ is a local minimum.
% 			\item If $\textrm{Dis}(a,b)<0$, then $f$ has a saddle point at $(a,b)$.
% 			\item If $\textrm{Dis}(a,b)=0$, then the test is inconclusive.
% 			\end{enumerate}
% 		\end{itemize}
		
% 	\item Absolute Extrema on Closed Bounded Regions (pg. 760)
	
% 		\begin{itemize}
% 		\item Let $f(p)$ be a function of many variables defined on a region containing the point $p_0$.
% 			\begin{itemize}
% 			\item If $f(p_0)\geq f(p)$ for all points $p$ in the domain of $f$, then the number $f(p_0)$ is called the \textbf{absolute maximum value} at $p_0$ of $f$. (That is, the value of $f$ is a local maximum if it is the largest value of the function.)
% 			\item If $f(p_0)\leq f(p)$ for all points $p$ in the domain of $f$, then the number $f(p_0)$ is called the \textbf{absolute minimum value} at $p_0$ of $f$. (That is, the value of $f$ is a local minimum if it is the smallest value of the function.)
% 			\end{itemize}
% 		\item It can be proven that, as it is for functions of a single variable, every function of many variables defined on a closed and bounded domain has absolute extrema on that domain.
% 		\item The first derivative test then shows that these absolute extrema must happen on either a critical point of the domain's interior or a boundary point of the domain.
% 		\end{itemize}
		
% 	\item Finding Absolute Extrema on a Closed Bounded Region (pg. 760)
	
% 		\begin{itemize}
% 		\item Follow these steps to find the absolute maximum and minimum value of a function $f(x,y)$ of two variables on a closed, bounded domain $D$.
% 			\begin{enumerate}
% 			\item Find the critical points on the interior of $D$.
% 			\item Take each portion of the boundary and define $g(x)$ or $g(y)$ to equal $f(x,y)$ on that boundary. Find the critical points of $g$ for each point on that boundary.
% 			\item Take any corners of $D$.
% 			\item Plug in all critical points and corners found into $f(x,y)$. The largest of these is the absolute maximum, and the smallest of these is the absolute minimum.
% 			\end{enumerate}
% 		\end{itemize}
		
% 	\item \textbf{Suggested Exercises for 12.7:}
	
% 		\begin{itemize}
% 		\item Finding local max/min and saddle points: 1-30
% 		\item Finding absolute max/min: 31-36
% 		\end{itemize}
		
% 	\end{itemize}
	
% 	\newpage
	
% 	\centerline{\bf 12.8 Lagrange Multipliers}
	
% 	In order to simplify the process for finding absolute extrema, we introduce (without proof) the method of Lagrange Multipliers.
	
% 	\begin{itemize}
	
% 	\item The Method of Lagrange Multipliers (pg. 769)
	
% 		\begin{itemize}
% 		\item The \textbf{Method of Lagrange Multipliers} says that if $f(p)$ is a function of many variables which has an absolute extreme value on the restricted domain $\{p\in \mathbb{R}^n:g(p)=c\}$, and $f,g$ are differentiable functions such that $\nabla g \not= \vect{0}$, then it occurs at a point $p_0$ such that \[\nabla f_{p_0}=\lambda \nabla g_{p_0} \text{ and } g(p_0)=c\] for some value of $\lambda\in\mathbb{R}$.
% 		\end{itemize}
		
% 	\item \textbf{Suggested Exercises for 12.8:}
	
% 		\begin{itemize}
% 		\item Finding absolute extrema using the Method of Lagrange Multipliers: 1-30
% 		\end{itemize}
		
% 	\end{itemize}
	
% 	\newpage
	
% 	\centerline{\bf 13.1 Double and Iterated Integrals over Rectangles}
	
% 	In this section, we extend the idea of a definite integral for multiple variables.
	
% 	\begin{itemize}
	
% 	\item Double Integral as Volume
	
% 		\begin{itemize}
% 		\item The definite integral \[\int\limits_I f(x)\, dx = \int_a^b f(x)\, dx\] over the interval $I=[a,b]\subseteq \mathbb{R}$ can be thought of as the net area (area above minus area below) between the curve given by $y=f(x)$ and the $x$-axis on that interval.
% 		\item Likewise, the double integral \[\iint_R f(x,y)\, dA\] over the rectangle $R \subseteq \mathbb{R}^2$ is defined to represent the net volume (volume above minus volume below) between the surface given by $z=f(x,y)$ and the $xy$-plane on that region.
% 		\item Let $f(x,y)$ be a continuous function of two variables and $R$ be the rectangle with corners $(a,c)$, $(b,c)$, $(b,d)$, and $(a,d)$, so that $a\leq x\leq b$ and $c\leq y\leq d$. Recalling Calculus II, to find $\iint_R f(x,y) dA$, we first find the cross-sectional net area \[A(y)=\int_{x=a}^{x=b} f(x,y) dx\] for an arbitrary value of $y$, and then integrate over $y\in[c,d]$ to get \[\iint_R f(x,y)\ dA = \int_{y=c}^{y=d} A(y)\ dy = \int_{y=c}^{y=d}\left(\int_{x=a}^{x=b} f(x,y)\ dx\right)dy\]
% 		\item We call this last integral an \textbf{iterated} or \textbf{repeated integral} and write it \[\int_c^d\int_a^b f(x,y)dxdy\]
% 		\item By using the cross-sectional area $A(x)=\int_{y=c}^{y=d} f(x,y) dy$, we may also find that \[\iint_R f(x,y) dA = \int_c^d\int_a^b f(x,y)dxdy = \int_a^b\int_c^d f(x,y)dydx\]
% 		\end{itemize}
		
% 	\item \textbf{Suggested Exercises for 13.1:}
	
% 		\begin{itemize}
% 		\item Evaluating iterated integrals: 1-12
% 		\item Evaluating double integrals: 13-28
% 		\end{itemize}
		
% 	\end{itemize}
	
% 	\newpage
	
% 	\centerline{\bf 13.2 Double Integrals over General Regions}
	
% 	In the previous section, we integrated over rectangles, which essentially was just the composition of two integrals over intervals. Now, we approach the goal of integrating over a general 2D region.
	
% 	\begin{itemize}
	
% 	\item Double Integral over Bounded Nonrectangular Regions
	
% 		\begin{itemize}
% 		\item As before, the double integral \[\iint_R f(x,y)\, dA\] over the bounded region $R \subseteq \mathbb{R}^2$ is defined to represent the net volume (volume above minus volume below) between the surface given by $z=f(x,y)$ and the $xy$-plane on that region.
% 		\item Let $R$ be a region in the $xy$-plane bounded by the functions $g_1(x)\leq g_2(x)$ and the vertical lines $x=a$ and $x=b$ for $a\leq b$.
% 		\item Let $f(x,y)$ be a continuous function. We can compute the cross-sectional net area for an arbitrary value of $x$ as \[A(x) = \int_{g_1(x)}^{g_2(x)} f(x,y)\, dy\]
% 		\item We may then integrate over $a\leq x\leq b$ to find the net volume \[\iint_R f(x,y)\, dA = \int_a^b A(x)\,dx \]\[= \int_a^b\left(\int_{g_1(x)}^{g_2(x)} f(x,y)\,dy\right)dx=\int_a^b\int_{g_1(x)}^{g_2(x)} f(x,y)\,dy\,dx\]
% 		\item Similarly, if $R$ is the region bounded by the functions $h_1(y)\leq h_2(y)$ and the horizontal lines $y=c$ and $y=d$ for $c\leq d$, then we may compute \[\iint\limits_R f(x,y) \,dA = \int_c^d\int_{h_1(y)}^{h_2(y)} f(x,y) \, dx \, dy\]
% 		\end{itemize}
% 	\newpage
% 	\item Finding Limits of Integration
	
% 		\begin{itemize}
% 		\item Often you'll have the region of integration $R$ described, and you need to convert this to limits of integration.
% 		\item Follow these steps:
% 			\begin{enumerate}
% 			\item Sketch the region of integration and label the bounding curves.
% 			\item Determine if it is easier to describe the top and bottom of the region as functions of $x$ or if it is easier to describe the left and right of the curve as functions of $y$.
% 				\begin{itemize}
% 				\item \textbf{If functions of $x$ are chosen:}
% 				\end{itemize}
% 			\item Find the $x$-limits of integration $a,b$ by finding the leftmost (for $a$) and rightmost (for $b$) $x$-values in the region.
% 			\item Find the $y$-limits of integration $g_1(x),g_2(x)$ by using the functions giving the bottom (for $g_1$) and top (for $g_2$) portions of the boundary.
% 			\item Plug these values into the integral \[\int_a^b\int_{g_1(x)}^{g_2(x)}f(x,y)\,dy\,dx\]
% 				\begin{itemize}
% 				\item \textbf{If functions of $y$ are chosen:}
% 				\end{itemize}
% 			\setcounter{enumi}{2}
% 			\item Find the $y$-limits of integration $c,d$ by finding the leftmost (for $c$) and rightmost (for $d$) $y$-values in the region.
% 			\item Find the $x$-limits of integration $h_1(y),h_2(y)$ by using the functions giving the left (for $h_1$) and right (for $h_2$) portions of the boundary.
% 			\item Plug these values into the integral \[\int_c^d\int_{h_1(y)}^{h_2(y)}f(x,y)\,dx\,dy\]
% 			\end{enumerate}
% 		\end{itemize}
	
% 	\item Swapping Variables of Integration
	
% 		\begin{itemize}
% 		\item Some double integrals may be difficult to evaluate using one particular order of integration (e.g. $\iint f \,dx\,dy$ versus $\iint f\,dy\,dx$).
% 		\item In that case, you can swap the order of integration by making an appropriate change of the limits of integration, using the above steps.
% 		\end{itemize}
% 	\newpage
% 	\item Properties of Double Integrals
	
% 		\begin{itemize}
% 		\item Let $f(x,y),g(x,y)$ be continuous.
% 			\begin{enumerate}
% 			\item Zero Integral
% 				\[\iint\limits_R 0\,dA = 0\]
% 			\item Constant Multiple
% 				\[\iint\limits_R cf(x,y)\,dA = c\iint\limits_R f(x,y)\,dA\]
% 			\item Sum/Difference
% 				\[\iint\limits_R f(x,y)\pm g(x,y)\,dA=\iint\limits_R f(x,y)\,dA\pm\iint\limits_R g(x,y)\,dA\]
% 			\item Domination
			
% 			If $f(x,y)\leq g(x,y)$ for all $(x,y)\in R$, then
% 				\[\iint\limits_R f(x,y)\,dA \leq \iint\limits_R g(x,y)\,dA\]
% 			\item Additivity
			
% 			If $R$ can be split into two regions $R_1,R_2$, then
% 				\[\iint\limits_R f(x,y)\,dA = \iint\limits_{R_1} f(x,y)\,dA + \iint\limits_{R_2} f(x,y)\,dA\]
% 			\end{enumerate}
% 		\end{itemize}
		
% 	\item \textbf{Suggested Exercises for 13.2:}
	
% 		\begin{itemize}
% 		\item Evaluating nonrectangular double integrals: 1-6, 11-14
% 		\item Finding limits of integration: 7-10, 33-44
% 		\item Swapping order of integration: 25-32
% 		\end{itemize}
		
% 	\end{itemize}
	
% 	\newpage
	
% 	\centerline{\bf 13.3 Area by Double Integration}
	
% 	Since the area of a region is equal to the volume of a cylinder of height $1$ unit with that region as its base, we have a new technique for computing the area of a region. We also may find a way to define the average value of a function of two variables.
	
% 	\begin{itemize}
	
% 	\item Areas of Bounded Regions in the Plane
		
% 		\begin{itemize}
% 		\item The area of a bounded region $R$ in the plane is given by \[\iint\limits_R\,dA=\iint\limits_R 1\,dA\]
% 		\end{itemize}
		
% 	\item Average Value of a Function of Two Variables
	
% 		\begin{itemize}
% 		\item Recall that the average value of a single variable function $f(x)$ over an interval $I=[a,b]$ is given by \[\frac{1}{b-a}\int_a^b f(x) \,dx = \frac{1}{\textrm{length of }I}\int_I f(x)\,dx\]
% 		\item We can extend this to functions of two variables by defining its average value over a bounded region $R$ to be \[\frac{1}{\textrm{area of }R}\iint\limits_R f(x,y)\,dA\]
% 		\end{itemize}
		
% 	\item \textbf{Suggested Exercises for 13.3:}
	
% 		\begin{itemize}
% 		\item Finding areas of regions: 1-8
% 		\item Finding average values of functions: 15-18
% 		\end{itemize}
		
% 	\end{itemize}
	
% 	\newpage
	
% 	\centerline{\bf 13.5 Triple Integrals in Rectangular Coordinates}
	
% 	\begin{itemize}
	
% 	\item Triple Integrals
		
% 		\begin{itemize}
% 		\item Just as length can be generalized into area, and area can be generalized into volume, we can generalize volume into ``hypervolume''.
% 		\item The triple integral \[\iiint\limits_D f(x,y,z)\, dV\] over the bounded region $D \subseteq \mathbb{R}^3$ is defined to represent the net ``hypervolume'' over the region.
% 		\end{itemize}
		
% 	\item Computing Triple Integrals
	
% 		\begin{itemize}
% 		\item Finding hypervolumes can be thought of as integrating over cross-sectional volumes - snapshots of the volume occurring when we fix one variable.
% 		\item Suppose $w=f(x,y,z)$, and let $D\subseteq \mathbb{R}^3$. If we fix $x$ in $w=f(x,y,z)$ and consider $f$ as a function of only $y,z$, the cross-sectional volume is given by the double integral \[V(x) = \iint\limits_{R(x)} f(x,y,z)\,dz\,dy\] where $R(z)$ is the cross-sectional area at the fixed value of $x$.
% 		\item Since the region $R(x)$ depends on $x$, its equivalent limits of integration are reliant on $x$ as well, giving \[V(x) = \int_{y=g_1(x)}^{y=g_2(x)}\int_{z=h_1(x,y)}^{z=h_2(x,y)}f(x,y,z)\,dz\,dy\]
% 		\item Finally, to find the hypervolume, we integrate $V(x)$ over the appropriate interval $[a,b]$ to find \[\iiint\limits_D f(x,y,z)\, dV=\int_{x_1}^{x_2}V(x) dt = \int_{x=a}^{x=b} \int_{y=g_1(x)}^{y=g_2(x)}\int_{z=h_1(x,y)}^{z=h_2(x,y)}f(x,y,z)\,dz\,dy\,dx\]
% 		\end{itemize}
% 	\newpage	
% 	\item Finding Limits of Integration
	
% 		\begin{itemize}
% 		\item To fill in the limits of integration, look for the following on your solid $D\subseteq \mathbb{R}^3$ of integration with a shadow $R=\{(x,y):(x,y,z)\in D\}\subseteq \mathbb{R}^2$:
% 			\begin{itemize}
% 			\item Functions $f_1(x,y),f_2(x,y)$ such that $z=f_1(x,y)$ is the bottom of the solid $D$ and $z=f_2(x,y)$ is the top of the solid $D$.
% 			\item Functions $g_1(x),g_2(x)$ such that $y=g_1(x)$ is the bottom of the region $R$ and $y=g_2(x)$ is the top of the region $R$.
% 			\item Numbers $a,b$ such that $x=a$ is on the leftmost point of $R$ and $x=b$ is on the rightmost point of $R$.
% 			\end{itemize}
% 		\item You may need to relabel $x,y,z$ to make this work.
% 		\end{itemize}
		
% 	\item Volume with Triple Integrals
	
% 		\begin{itemize}
% 		\item Similar to area, the volume of a 3D solid $D$ can be computed by the triple integral \[\iiint\limits_D\, dV \]
% 		\end{itemize}
		
% 	\item Average Value of a Three-Variable Function
	
% 		\begin{itemize}
% 		\item Similar to functions of one and two variables, we define the average value of $F(x,y,z)$ over the closed bounded solid $D\subseteq \mathbb{R}^3$ to be \[\frac{1}{\text{volume of } D} \iiint\limits_D F(x,y,z)\, dV\]
% 		\end{itemize}
		
% 	\item Triple Integral Properties
	
% 		\begin{itemize}
% 		\item The analogues of the properties for double integrals in Section 13.2 also hold for triple integrals.
% 		\end{itemize}
				
% 	\item \textbf{Suggested Exercises for 13.5:}
	
% 		\begin{itemize}
% 		\item Evaluating triple integrals: 7-20
% 		\item Finding volumes of solids: 23-36
% 		\item Finding the average value of functions: 37-40
% 		\end{itemize}
		
% 	\end{itemize}
	
% 	\newpage
	
% 	\centerline{\bf 13.8 Substitution in Multiple Integrals}
	
% 	\begin{itemize}
	
% 	\item Transformations
	
% 		\begin{itemize}
% 		\item Two similar regions in 2D space can be related by a ``nice'' pair of functions $x(u,v)$ and $y(u,v)$ that map points in a $uv$ plane to the $xy$ plane.
% 		\item Similarly, two similar solids in 3D space can be identified by a transformation given by three ``nice'' functions $x(u,v,w)$, $y(u,v,w)$, $z(u,v,w)$ that map points in a $uvw$ space to the $xyz$ space.
% 		\end{itemize}
	
% 	\item The Jacobian
	
% 		\begin{itemize}
% 		\item The Jacobian of a 2D transformation given by $x(u,v)$, $y(u,v)$ is the determinant
% \[
% J(u,v) = \frac{\partial (x,y)}{\partial (u,v)} =
% \begin{array}{|c c|}
% \frac{\partial x}{\partial u} & \frac{\partial x}{\partial v} \\
% \frac{\partial y}{\partial u} & \frac{\partial y}{\partial v} \\
% \end{array}
% \]
% 		\item The Jacobian of a 3D transformation given by $x(u,v,w)$, $y(u,v,w)$, $z(u,v,w)$ is the determinant
% \[
% J(u,v,w) = \frac{\partial (x,y,z)}{\partial (u,v,w)} =
% \begin{array}{|c c c|}
% \frac{\partial x}{\partial u} & \frac{\partial x}{\partial v} & \frac{\partial x}{\partial w} \\
% \frac{\partial y}{\partial u} & \frac{\partial y}{\partial v} & \frac{\partial y}{\partial w} \\
% \frac{\partial z}{\partial u} & \frac{\partial z}{\partial v} & \frac{\partial z}{\partial w} \\
% \end{array}
% \]
% 		\end{itemize}
		
% 	\item 2D Substitution
	
% 		\begin{itemize}
% 		\item Suppose that the region $R$ in the $xy$-plane is the result of applying the pair of functions $x(u,v)$, $y(u,v)$ to the region $G$ in the $uv$-plane.
% 		\item Then it follows that \[\iint\limits_R f(x,y)\,dx\,dy = \iint\limits_G f(x(u,v),y(u,v))|J(u,v)|\,du\,dv\]
% 		\item By applying the formula above, you can get a new integral with simpler limits of integration, or a simpler integrand.
% 		\end{itemize}
		
% 	\item 3D Substitution
	
% 		\begin{itemize}
% 		\item Suppose that the region $R$ in $xyz$ space is the result of applying the transformation $x(u,v,w)$, $y(u,v,w)$, $z(u,v,w)$ to the region $G$ in $uvw$ space.
% 		\item Then it follows that \[\iiint\limits_R f(x,y,z)\,dx\,dy\,dz \]\[= \iiint\limits_G f(x(u,v,w),y(u,v,w),z(u,v,w))|J(u,v,w)|\,du\,dv\,dw\]
% 		\item By applying the formula above, you can get a new integral with simpler limits of integration, or a simpler integrand.
% 		\end{itemize}
				
% 	\item \textbf{Suggested Exercises for 13.8:}
	
% 		\begin{itemize}
% 		\item 2D Jacobians, Transformations, and substitutions: 1-10
% 		\end{itemize}
		
% 	\end{itemize}
	
% 	\newpage
	
% 	\centerline{\bf 13.4 Double Integrals in Polar Form}
	
% 	\begin{itemize}
	
% 	\item Integrating over Regions expressed using Polar Coordinates
		
% 		\begin{itemize}
% 		\item Using the change of variables $x(r,\theta)=r\cos\theta$ and $y(r,\theta)=r\sin\theta$, we may integrate \[\iint\limits_R f(x,y)\, dA = \iint\limits_G f(r\cos\theta,r\sin\theta)\,r\dvar{r}\dvar{\theta}\]
% 		\end{itemize}
			
% 	\item \textbf{Suggested Exercises for 13.4:}
	
% 		\begin{itemize}
% 		\item Changing Cartesian integrals to polar integrals: 1-16
% 		\item Finding integrals over polar regions: 17-22
% 		\end{itemize}
		
% 	\end{itemize}
	
% 	\newpage
	
% 	\centerline{\bf 13.7 Triple Integrals in Cylindrical and Spherical Coordinates}
	
% 	\begin{itemize}
	
% 	\item Cylindrical Coordinates
% 		\begin{itemize}
% 		\item Using the change of variables $x(r,\theta,z)=r\cos\theta$, $y(r,\theta,z)=r\sin\theta$, and $z(r,\theta,z)=z$, we may integrate \[\iiint\limits_D F(x,y,z)\, dV = \iiint\limits_G F(r\cos\theta,r\sin\theta,z)\,r\dvar{z}\dvar{r}\dvar{\theta}\]
% 		\end{itemize}
	
% 	\item Spherical Coordinates
% 		\begin{itemize}
% 		\item Using the change of variables $x(\rho,\phi,\theta)=\rho\sin\phi\cos\theta$, $y(\rho,\phi,\theta)=\rho\sin\phi\sin\theta$, and $z(\rho,\phi,\theta)=\rho\cos\phi$, we may integrate \[\iiint\limits_D F(x,y,z)\, dV = \iiint\limits_G F(\rho\sin\phi\cos\theta,\rho\sin\phi\sin\theta,\rho\cos\phi) \,\rho^2\sin\phi\dvar{\rho}\dvar{\phi}\dvar{\theta} \]
% 		\end{itemize}
			
% 	\item \textbf{Suggested Exercises for 13.7:}
	
% 		\begin{itemize}
% 		\item Cylindrical coordinate integrals: 1-20
% 		\item Finding integrals over polar regions: 21-38
% 		\end{itemize}
	
% 	\end{itemize}
	
% 	\newpage
	
% 	\centerline{\bf 14.1 Line Integrals}
	
% 	A function of three variables evaluated over a curve in 3D space behaves like a function of one variable evaluated over the real line. In this section we investigate how to evaluate line integrals over such curves.
	
% 	\begin{itemize}
	
% 	\item Line Integrals with Respect to Arclength
	
% 		\begin{itemize}
% 		\item We may define \[\int\limits_C f(x,y,z)\dvar{s}\] to be the \textbf{line integral over a curve $C$}, which represents the area of a ``ribbon'' with a base at the curve $C$ and thickness given by $f(x,y,z)$ for each point $(x,y,z)\in C$.
% 		\item Suppose a curve $C$ in $\mathbb{R}^3$ is defined by the smooth vector function $\vect{r}(s)=\<x(s),y(s),z(s)\>$ for $a\leq s\leq b$.
% 		\item It follows then that \[\int\limits_C f(x,y,z)\dvar{s} = \int_{s=a}^{s=b} f(x(s),y(s),z(s))\dvar{s}\]
% 		\item Now suppose a curve $C$ in $\mathbb{R}^3$ is defined by the smooth vector function $\vect{r}(t)=\<x(t),y(t),z(t)\>$ for $a\leq t\leq b$.
% 		\item We may then use \[\int\limits_C f(x,y,z)\dvar{s}=\int_{t=a}^{t=b} f(x(t),y(t),z(t))|\vect{v}(t)|\dvar{t}\]
% 		\end{itemize}
		
% 	\item Additivity
	
% 		\begin{itemize}
% 		\item If $C$ can be partitioned into two curves $C_1$ and $C_2$, then \[\int\limits_C f\dvar{s}=\int\limits_{C_1}f\dvar{s}+\int\limits_{C_2}f\dvar{s}\]
% 		\end{itemize}
		
% 	\newpage
	
% 	\item Reversing Curves
	
% 		\begin{itemize}
% 		\item If $C$ and $-C$ designate the same curve oriented in opposite directions, then \[\int\limits_{C} f\dvar{s} = \int\limits_{-C} f\dvar{s}\]
% 		\end{itemize}
				
% 	\item \textbf{Suggested Exercises for 14.1:}
	
% 		\begin{itemize}
% 		\item Identifying vector equations for graphs: 1-8
% 		\item Evaluating line integrals: 9-22
% 		\end{itemize}
		
% 	\end{itemize}
	
% 	\newpage
	
% 	\centerline{\bf 14.2 Vector Fields, Work, Circulation, and Flux}
	
% 	\begin{itemize}
	
% 	\item Line Integrals with Respect to Variables
	
% 		\begin{itemize}
% 			\item Often it will be useful to consider \textbf{line integrals with respect to $x$, $y$, or $z$ over a directed curve $C$}.
% 			\item These integrals, of the form \[\int\limits_C f(x,y,z)\,dx\] (or $dy$ or $dz$) may be computed by parametrizing $C$ with a smooth vector function $\vect{r}(t)=\<x(t),y(t),z(t)\>$ for $a\leq t\leq b$ and evaluating \[\int\limits_C f(x,y,z)\dvar{x} = \int_{t=a}^{t=b} f(x(t),y(t),z(t))\,\frac{dx}{dt}\dvar{t}\]
% 			\item Such integrals have the property \[\int\limits_{-C} f\dvar{x} = -\int\limits_{C} f\dvar{x}\]
% 		\end{itemize}
	
% 	\item Vector Fields
	
% 		\begin{itemize}
% 		\item A \textbf{vector field} is a function $\vect{F}(x,y,z)=\<M(x,y,z),N(x,y,z),P(x,y,z)\>$ which assigns a vector to each point in its domain.
% 		\item We've seen an important example of a vector field, the \textbf{gradient field} $\nabla f = \<\frac{\partial f}{\partial x}, \frac{\partial f}{\partial y},\frac{\partial f}{\partial z}\>$.
% 		\end{itemize}
	
% 	\item Line Integrals over Vector Fields
	
% 		\begin{itemize}
% 		\item We often wish to compute the \textbf{line integral of a vector field over a directed curve $C$}, using \[\int\limits_C \vect{F}\cdot d\vect{r} = \int\limits_C M(x,y,z)\,dx + \int\limits_C N(x,y,z)\,dx + \int\limits_C P(x,y,z)\,dz\]
% 		\item The notation $\ds \int\limits_C M\,dx + N\,dy + P\,dz$ is often used for short.
% 		\item It follows that \[\int\limits_{-C} \vect{F}\cdot d\vect{r} = - \int\limits_C \vect{F}\cdot d\vect{r}\]
% 		\end{itemize}
		
% 	\item Work over a Smooth Curve
	
% 		\begin{itemize}
% 		\item Recall that the work done over a displacement vector $\vect{D}=\<d_x,d_y,d_z\>$ by a force $\vect{F}=\<M,N,P\>$ is \[W = \vect{F} \cdot \vect{D}=Md_x+Nd_y+Pd_z\]
% 		\item So by subdividing a portion of a smooth curve $C$ with vector function $\vect{r}(t)$ over $t\in[a,b]$ by a force $\vect{F}(x,y,z)$, we find an approximation of the work done by the sum \[W \approx \sum_{i=1}^n \vect{F}(x_i,y_i,z_i)\cdot\Delta\vect{r_i}\]\[ = \sum_{i=1}^n \left(M(x_i,y_i,z_i)\Delta x_i + N(x_i,y_i,z_i)\Delta y_i + P(x_i,y_i,z_i)\Delta z_i\right) \]
% 		\item Limiting this Riemann sum to infinity results in the definition \[W = \int\limits_C \vect{F}\cdot d\vect{r} = \int\limits_C M\,dx+N\,dy+P\,dz\]
% 		\item This may also be computed using \[W = \int\limits_C M\,dx+N\,dy+P\,dz = \int\limits_C \vect{F}\cdot\vect{v}\,dt = \int\limits_C \vect{F}\cdot \vect{T}\,ds\]
% 		\end{itemize}
		
% 	\item Flow
	
% 		\begin{itemize}
% 		\item The \textbf{flow} of a fluid flowing through a curve in space given by $\vect{r}(t)$ on $t\in[a,b]$ is defined to be the integral \[\textrm{Flow} = \int\limits_C \vect{F}\cdot d\vect{r} = \int\limits_C \vect{F}\cdot\vect{v}\,dt = \int\limits_C \vect{F}\cdot \vect{T}\,ds\] where $\vect{F}$ is the velocity fluid of the fluid.
% 		\item If $C$ is closed (its starting point and ending point are the same), then the flow is also known as the \textbf{circulation}.
% 		\end{itemize}
		
% 	\newpage
		
% 	\item Flux
	
% 		\begin{itemize}
% 		\item If $\vect{n}(x,y)$ is the outward unit vector normal to a closed plane curve C at $(x,y)$ and $\vect{F}(x,y)$ is a planar vector field, the \textbf{flux} of $\vect{F}$ across $C$ is \[\int\limits_C \vect{F}\cdot\vect{n}\,ds\]
% 		\item If $\vect{F}(x,y)=\<M,N\>$ and the direction traveled around $C$ is counter-clockwise, then \[\int\limits_C \vect{F}\cdot\vect{n}\dvar{s} = \int\limits_C \vect{F}\cdot(\veck\times\vect{T})\dvar{s} \]\[= \int\limits_C \<M,N\>\cdot\<\frac{dy}{ds},-\frac{dx}{ds}\>\dvar{s} = \int\limits_C M\dvar{y} - N\dvar{x}\]
% 		\end{itemize}
				
% 	\item \textbf{Suggested Exercises for 14.2:}
	
% 		\begin{itemize}
% 		\item Work over a curve: 7-22
% 		\item Circulation, flow, and flux: 23-28, 37-40
% 		\end{itemize}
		
% 	\end{itemize}
	
% 	\newpage
	
% 	\centerline{\bf 14.3 Path Independence, Potential Functions, and Conservative Fields}
	
% 	\begin{itemize}
		
% 	\item Technical Assumptions on Curves and Regions for this section
	
% 		\begin{itemize}
% 		\item We make certain assumptions on curves, fields, and regions in this section, which are required for the results to hold.
% 			\begin{itemize}
% 			\item All curves are \textbf{piecewise smooth}: they are composed of finite smooth pieces joined end-to-end.
% 			\item All vector fields have components with continuous first partial derivatives.
% 			\item Regions $D$ are \textbf{simply connected}: a simply connected region is a single piece with no holes.
% 			\end{itemize}
% 		\end{itemize}
		
% 	\item Several Equivalencies for Conservative Fields
% 		\begin{itemize}
% 		\item \textbf{The following are all equivalent:}
% 			\begin{itemize}
% 			\item $\vect{F}=\<M,N,P\>$ is a \textbf{conservative field} on $D$.
% 			\item $\int\vect{F}\cdot d\vect{r}$ is \textbf{path independent} in $D$.
% 				\begin{itemize}
% 				\item This means that the value of $\int_C\vect{F}\cdot d\vect{r}$ only depends on the endpoints of the curve $C$.
% 				\end{itemize}
% 			\item There exists a \textbf{potential function} $f$ for $\vect{F}$.
% 				\begin{itemize}
% 				\item This means that $\nabla f = \vect{F}$.
% 				\end{itemize}
% 			\item (Closed Loop Property of Conservative Fields)\newline $\ds \int_C \vect{F} \cdot d\vect{r} = 0$ for every closed loop $C$ in $D$. 
% 			\item (Fundamental Theorem of Line Integrals)\newline $\ds \int_C \vect{F} \cdot d\vect{r} = f(B)-f(A)$ for every path $C$ in $D$ connecting $A$ to $B$. 
% 			\item $M\,dx+N\,dy+P\,dz$ is \textbf{exact}.
% 				\begin{itemize}
% 				\item This means that there exists a function $f$ such that $M\,dx+N\,dy+P\,dz = f_x\,dx+f_y\,dy+f_z\,dz$.
% 				\end{itemize}
% 			\item (Component Test for Conservative Fields)\newline $\ds \frac{\partial P}{\partial y}=\frac{\partial N}{\partial z},\,\frac{\partial M}{\partial z}=\frac{\partial P}{\partial x},\text{ and }\frac{\partial N}{\partial x}=\frac{\partial N}{\partial y}$. 
% 			\end{itemize}
% 		\end{itemize}
% 	\newpage			
% 	\item \textbf{Suggested Exercises for 14.3:}
	
% 		\begin{itemize}
% 		\item Determining if a field is conservative: 1-6
% 		\item Finding potential functions: 7-12
% 		\item Evaluating integrals of differential forms: 13-22
% 		\end{itemize}
		
% 	\end{itemize}
	
% 	\newpage
	
% 	\centerline{\bf 14.4 Green's Theorem in the Plane}
	
% 	\begin{itemize}
		
% 	\item Gradient Operator
	
% 		\begin{itemize}
% 		\item Recall that the gradient vector is defined to be \[\nabla f = \<\frac{\partial f}{\partial x},\frac{\partial f}{\partial y},\frac{\partial f}{\partial z}\>\]
% 		\item We may also think of it as the scalar multiplication of $f$ with the \textbf{gradient operator} \[\nabla = \<\frac{\partial}{\partial x},\frac{\partial}{\partial y},\frac{\partial}{\partial z}\>\]
% 		\end{itemize}
		
% 	\item Divergence
	
% 		\begin{itemize}
% 		\item The \textbf{divergence} of a planar vector field $\vect{F}=\<M,N\>$ is given by \[ \div \vect{F} = \frac{\partial M}{\partial x}+\frac{\partial N}{\partial y} = \nabla\cdot\vect{F} \]
% 		\item Intuitively, divergence measures the tendency of ``nearby'' vectors in the field pointing away from the point.
% 		\item In physics, divergence is often called the \textbf{flux density}.
% 		\end{itemize}
	
% 	\item Spin
	
% 		\begin{itemize}
% 		\item The \textbf{spin} of a planar vector field $\vect{F}=\<M,N\>$ is given by \[ \spin \vect{F} = \frac{\partial N}{\partial x}-\frac{\partial M}{\partial y} \]
% 		\item Intuitively, spin measures the tendency of ``nearby'' vectors in the field to turn counter-clockwise around the point.
% 		\item In physics, spin is often called the \textbf{circulation density}.
% 		\item Spin is also the \textbf{$\veck$-component of curl}, defined in a later section.
% 		\end{itemize}
		
% 	\item Simple Curves
% 		\begin{itemize}
% 		\item A curve which does not cross itself is said to be \textbf{simple}.
% 		\end{itemize}
% 	\newpage
% 	\item Green's Theorem in the Plane
% 		\begin{itemize}
% 		\item There are two forms of Green's Theorem. They both start the same way:
% 		\item Let $C$ be a piecewise smooth, simple closed curve enclosing the region $R$ and oriented counter-clockwise. Let $\vect{F}=\<M,N\>$ be a vector field for which $M,N$ have continuous first partial derivatives in an open region containing $R$.
% 		\item \textbf{Flux-Divergence Form}
% 			\begin{itemize}
% 			\item The flux across $C$ equals the double integral of the divergence of $\vect{F}$ over $R$. That is, \[\int\limits_C \vect{F}\cdot\vect{n}\,ds = \iint\limits_R \div \vect{F}\dvar{A}\]
% 			\item Intuitively, this is true because the total flux measuring how vectors leave the curve is related to the total divergence of vectors within the curve.
% 			\end{itemize}
% 		\item \textbf{Circulation-Spin or Circulation-Curl Form}
% 			\begin{itemize}
% 			\item The counter-clockwise circulation around $C$ equals the double integral of the spin of $\vect{F}$ over $R$. That is, \[\int\limits_C \vect{F}\cdot\vect{T}\dvar{s} = \iint\limits_R \spin \vect{F}\dvar{A}\]
% 			\item Intuitively, this is true because the total circulation measuring how vectors traverse the curve counter-clockwise is related to the total counter-clockwise spin of vectors within the curve.
% 			\end{itemize}
% 		\end{itemize}
		
% 	\item \textbf{Suggested Exercises for 14.4:}
	
% 		\begin{itemize}
% 		\item Using Green's Theorem to find circulation and flux: 5-14
% 		\item Using Green's Theorem to evaluate line integrals: 17-20
% 		\end{itemize}
		
% 	\end{itemize}
	
% 	\newpage
	
% 	\centerline{\bf 14.5 Surfaces and Area}
	
% 	\begin{itemize}
		
% 	\item Parametrization of Surfaces
	
% 		\begin{itemize}
% 		\item Just as we can define a vector function \[\vect{r}(t)=\<x(t),y(t),z(t)\>\] to describe a curve in space, we may define a vector function \[\vect{r}(u,v)=\<x(u,v),y(u,v),z(u,v)\>\] to describe a surface in space.
% 		\item The functions $x(u,v)$, $y(u,v)$, $z(u,v)$ are said to \textbf{parametrize} the surface.
% 		\end{itemize}
		
% 	\item Vector Function Partial Derivatives and Smooth Vector Functions
	
% 		\begin{itemize}
% 		\item The partial derivative $\vect{r}_u=\frac{\partial \vect{r}}{\partial u}$ of the vector function $\vect{r}(u,v)$ is given by \[\vect{r}_u=\<x_u,y_u,z_u\>\]
% 		\item Similarly, \[\vect{r}_v=\<x_v,y_v,z_v\>\]
% 		\item A surface parametrized by $\vect{r}(u,v)$ is called \textbf{smooth} if $\vect{r}_u,$ $\vect{r}_v$ are continuous and $\vect{r}_u\times\vect{r}_v\not= \vect{0}$ on the interior of the surface. 
% 		\end{itemize}
		
% 	\item Surface Area of a Parametrized Surface
	
% 		\begin{itemize}
% 		\item The area of a smooth surface with parametrizing vector function $\vect{r}(u,v)$ for a region $R$ in the $uv$ plane is given by \[A = \iint\limits_R |\vect{r}_u\times\vect{r}_v|\,dA\]
% 		\end{itemize}
		
% 	\item Implicit Surface
	
% 		\begin{itemize}
% 		\item Level surfaces $F(x,y,z)=c$ are sometimes called \textbf{implicit surfaces} because they don't always have a nice parametrization.
% 		\item It can be found that, if $\vect{p}$ is a unit vector normal a coordinate plane, then the surface area defined by $F(x,y,z)$ bounded by the cylinder given by a region $R$ in that coordinate plane is \[\iint\limits_R \frac{|\nabla F|}{|\nabla F \cdot \vect{p}|}\,dA\]
% 		\end{itemize}
		
% 	\item Surface Area Differential
	
% 		\begin{itemize}
% 		\item The integral $\iint\limits_S\, d\sigma$ is used to represent surface area, and $d\sigma$ is known as the surface area differential.
% 		\item Thus we have, for parametrized surfaces given by $\vect{r}(u,v)$: \[d\sigma = |\vect{r}_u\times\vect{r}_v|\,dA\] and for implicit surfaces given by a  level surface $F(x,y,z)=c$: \[d\sigma = \frac{|\nabla F|}{|\nabla F \cdot \vect{p}|}\,dA\]
% 		\end{itemize}
				
% 	\item \textbf{Suggested Exercises for 14.5:}
	
% 		\begin{itemize}
% 		\item Finding parametrizations of surfaces: 1-16
% 		\item Finding surface area: 17-26
% 		\end{itemize}
		
% 	\end{itemize}
	
% 	\newpage
	
% 	\centerline{\bf 14.6 Surface Integrals and Flux}
	
% 	\begin{itemize}
		
% 	\item Surface Integrals
	
% 		\begin{itemize}
% 		\item The \textbf{surface integral} of a function $G(x,y,z)$ over a surface $S$ is given by \[\iint\limits_S G(x,y,z)\,d\sigma\]
		
% 		\item This integral may be computed by parametrizing $S$ with \[\vect{r}(u,v)=\<x(u,v),y(u,v),z(u,v)\>\] for $(u,v)\in R$ and evaluating \[\iint\limits_S G(x,y,z)\,d\sigma = \iint\limits_R G(x(u,v),y(u,v),z(u,v)) |\vect{r}_u\times \vect{r}_v|\,dA \]
		
% 		\item Or, if $S$ is given by $F(x,y,z)=c$ with a shadow $R$ in a coordinate plane normal to the unit vector $\vect{p}$, the surface integral can be evaluated using \[\iint\limits_S G(x,y,z)\,d\sigma = \iint\limits_R G(x,y,z)\frac{|\nabla F|}{|\nabla F \cdot \vect{p}|}\,dA\]
% 		\end{itemize}
		
% 	\item Orientable Surfaces
% 		\begin{itemize}
% 		\item A surface is said to be \textbf{orientable} if it is ``two-sided''. More technically, it is orientable if there exists a continuous normal unit vector field $\vect{n}$ to the surface.
% 		\item An real-life example of a non-orientable surface is the Mobius strip formed by twisting a strip of paper together once and taping its ends together.
% 		\end{itemize}
		
% 	\item Flux in Three Dimensions
	
% 		\begin{itemize}
% 		\item The flux of a three dimensional vector field $\vect{F}$ across an oriented surface $S$ in the direction of $\vect{n}$ is given by the surface integral \[\iint\limits_S \vect{F}\cdot\vect{n}\,d\sigma\]
% 		\end{itemize}
				
% 	\item \textbf{Suggested Exercises for 14.6:}
	
% 		\begin{itemize}
% 		\item Evaluating surface integrals: 1-14
% 		\item Three-dimensional flux: 15-24
% 		\end{itemize}
		
% 	\end{itemize}
	
% 	\newpage
	
% 	\centerline{\bf 14.7 Stokes' Theorem}
	
% 	\begin{itemize}
		
% 	\item Curl
% 		\begin{itemize}
% 		\item The \textbf{curl} of a vector field $\vect{F}$ is defined as \[\textrm{curl}\,\vect{F} = \nabla \times \vect{F}\]
% 		\item Expanding this cross product, we see \[\textrm{curl}\,\vect{F} = \<\frac{\partial P}{\partial y}-\frac{\partial N}{\partial z},\frac{\partial M}{\partial z}-\frac{\partial P}{\partial x},\frac{\partial N}{\partial x}-\frac{\partial M}{\partial y}\>\]
% 		\item Recalling that, for a vector field in the $xy$ plane ($z=0$), \[\textrm{spin}\,\vect{F} = \frac{\partial N}{\partial x}-\frac{\partial M}{\partial y}\] we see that the curl vector measures the vector field's spin about that point on the planes parallel to $x=0$, $y=0$, and $z=0$ respectively.
% 		\end{itemize}
	
% 	\item Stokes' Theorem
	
% 		\begin{itemize}
% 		\item Recall that the counter-clockwise circulation about a curve $C$ in the plane bounding the region $R$ can be computed by \[\int\limits_C \vect{F}\cdot\vect{T}\,ds = \iint\limits_R \textrm{spin}\,\vect{F}\,dA\]
% 		\item Noting that in $\mathbb{R}^2$ \[\textrm{spin}\,\vect{F} = \textrm{curl}\,\vect{F}\cdot \veck = \nabla \times \vect{F} \cdot \veck\] in $\mathbb{R}^3$ we may define the counterclockwise spin with respect to the vector $\vect{v}$ to be \[\textrm{spin}_{\vect{v}}\,\vect{F} = \textrm{curl}\,\vect{F} \cdot \vect{v} = \nabla \times \vect{F} \cdot \vect{v}\]
% 		\item If a curve $C$ in $\mathbb{R}^3$ is the boundary of a surface $S$, and we want to compute the counter-clockwise circulation with respect to unit normal vectors $\vect{n}$ on the surface, we may use \[\int\limits_C \vect{F}\cdot\vect{T}\,ds = \iint\limits_S \textrm{spin}_{\vect{n}}\,\vect{F}\,d\sigma = \iint\limits_S (\textrm{curl}\,\vect{F} \cdot \vect{n})\,d\sigma = \iint\limits_S \nabla \times \vect{F} \cdot \vect{n}\,d\sigma\] 
% 		\end{itemize}
		
% 	\item Identities and Properties
	
% 		\begin{itemize}
% 		\item Due to the Mixed Derivative Theorem, \[\textrm{curl}\, \nabla f = \nabla \times \nabla f = \vect{0}\]
% 		\item If $\nabla \times \vect{F} = \vect{0}$ for every point in a region $D$, then \[ \int\limits_C \vect{F}\cdot d\vect{r} = \iint\limits_S \nabla \times \vect{F} \cdot \vect{n}\,d\sigma = 0\] for every curve $C$ and surface $S$ within $D$.
% 		\end{itemize}
				
% 	\item \textbf{Suggested Exercises for 14.7:}
	
% 		\begin{itemize}
% 		\item Using Stokes' Theorem: 1-10
% 		\end{itemize}
		
% 	\end{itemize}
	
% 	\newpage
	
% 	\centerline{\bf 14.8 Divergence Theorem and a Unified Theory}
	
% 	\begin{itemize}
		
% 	\item Divergence Theorem
	
% 		\begin{itemize}
% 		\item Divergence in $\mathbb{R}^2$ was defined as \[ \textrm{div}\,\vect{F} = \frac{\partial M}{\partial x}+\frac{\partial N}{\partial y} = \nabla \cdot \vect{F} \] and is defined in $\mathbb{R}^3$ as \[\textrm{div}\,\vect{F} = \frac{\partial M}{\partial x}+\frac{\partial N}{\partial y}+\frac{\partial P}{\partial z} = \nabla \cdot \vect{F} \]
% 		\item In both cases it measures the tendency of the vector field to point outward from a point.
% 		\item The Divergence Theorem lets us measure the flux on a closed surface $S$ by integrating over the divergence within its bounded region $D$: \[\text{Flux} = \iint\limits_S \vect{F}\cdot\vect{n}\,d\sigma = \iiint\limits_D \textrm{div}\,\vect{F}\,dV= \iiint\limits_D \nabla\cdot\vect{F}\,dV\]
% 		\end{itemize}
		
% 	\item The Unified Theory
	
% 		\begin{itemize}
% 		\item The unified theory notes that in order to compute circulation and flux over a closed curve or surface, we may consider the spin/curl and divergence over the region bounded by that curve or surface.
% 		\item Let $C$ be a counter-clockwise closed curve in $\mathbb{R}^2$ bounding the region $R$.
% 			\[\text{Circulation of } \vect{F} \text{ around } C = \iint\limits_R \textrm{spin}\,\vect{F}\,dA = \iint\limits_R \textrm{curl}\,\vect{F}\cdot \veck\,dA\]
% 			\[\text{Flux of } \vect{F} \text{ across } C = \iint\limits_R \textrm{div}\,\vect{F}\,dA\]
% 		\item Let $C$ be a closed curve in $\mathbb{R}^3$ counter-clockwise to $\vect{n}$ bounding the surface $S$.
% 			\[\text{Circulation of } \vect{F} \text{ around } C = \iint\limits_S \textrm{spin}_{\vect{n}}\,\vect{F}\,d\sigma  = \iint\limits_R \textrm{curl}\,\vect{F}\cdot \vect{n}\,d\sigma\]
% 		\item Let $S$ be a closed surface in $\mathbb{R}^3$ bounding the solid $D$.
% 			\[\text{Flux of } \vect{F} \text{ across } S = \iiint\limits_D \textrm{div}\,\vect{F}\,dV\]
% 		\end{itemize}
				
% 	\item \textbf{Suggested Exercises for 14.8:}
	
% 		\begin{itemize}
% 		\item Using the Divergence Theorem: 5-16
% 		\end{itemize}
		
% 	\end{itemize}

\end{document}





















