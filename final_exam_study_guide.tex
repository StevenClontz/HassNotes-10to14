\documentclass[12pt]{article}
\pagestyle{empty}

\usepackage{graphicx}
\usepackage{amsmath}

\setlength\oddsidemargin{-0.5in}
\setlength\evensidemargin{-0.5in}
\setlength\textwidth{7in}
\setlength\textheight{10in}
\setlength\topmargin{0in}
\setlength\headheight{0in}
\setlength\headsep{0in}

\newcommand{\up}{$~$\vspace*{-0.7in}}
\newcommand{\ans}{\noindent Ans.\underline{\hspace{3in}}}
\newcommand{\liner}{\noindent\underline{\hspace*{7in}}}
\newcommand{\spac}{\hspace*{3em}}
\newcommand{\ds}{\displaystyle}
\newcommand{\halfspac}{\hspace*{1.5em}}

\begin{document}

\up

{\bf Final Study Guide  \hspace*{1em} Your Name: \_\_\_\_\_\_\_\_\_\_\_\_\_\_\_\_\_\_\_\_\_\_\_\_\_\_ \hspace*{1em} Class: 9am / 1pm}

\vspace*{0.2in}

\centerline{ \bf Calculus III - Math 2630 - Spring 2013 \spac Instructor: Steven Clontz}

\vspace*{0.2in}

\centerline{\bf Draw a \framebox{box} around your answer. Show your work. Calculators not allowed.}

\indent\liner

\vspace*{.2in}

The first 12 questions are based upon the previous four tests, covering Chapters 10-13, based upon three questions from each test. Use these questions (and similar ones from the textbook) as your study guide:

\begin{itemize}
\item Ch 10 Test/Study Guide: \#5, \#6, \#10
\item Ch 11 Test/Study Guide: \#3, \#6, \#8
\item Ch 12 Test/Study Guide: \#3, \#5, \#8
\item Ch 13 Test/Study Guide: \#2, \#5, \#9
\end{itemize}

The last 4 questions are based upon the Chapter 14 material, and will be based upon some subset of the questions on the following pages.

\newpage\up


\begin{enumerate}

% (14.1) Line Integral
\item Evaluate \[ \int\limits_C xy^3\,ds\] where $C$ is the arc on the circle $x^2+y^2=4$ oriented clockwise from $(0,2)$ to $(\sqrt{3},1)$.

\vspace*{8in}

\liner

\newpage\up

% (14.2) Work Line Integral

\item Compute the work done by the force \[\vec{F}=\left<y,z,x\right>\] over the line segment from $(1,1,2)$ to $(3,-2,1)$.

\vspace*{8.5in}

\liner

\newpage\up

% (14.3) Compute work/flow of a conservative field.

\item Compute the flow of the vector field \[\vec{F}=\left<2xy,x^2-z^2,-2yz\right>\] through the curve $\vec{r}(t)=\left<t2^t,3t^3,\cos(\pi t)\right>$ where $0\leq t\leq 1$. (Hint: Use a potential function.)

\vspace*{8.5in}

\liner

\newpage\up

% (14.3) Line integral of conservative field over a closed loop.

\item Show that \[\int\limits_C (ye^{xy}-4yz)\,dx+(xe^{xy}-4xz)\,dy+(-4xy)\,dz = 0\] where $C$ is the pentagon in the $xz$ plane with vertices $(1,0,0)$, $(2,0,1)$, $(2,0,3)$, $(0,0,2)$, and $(0,0,0)$ oriented clockwise with respect to the $y$-axis.

\vspace*{8in}

\liner

\newpage\up

% (14.4) Green's Theorem for Flux

\item Express the outward flux of \[\vec{F}=\left<x+y,x^2+y^2\right>\] across the triangle with vertices $(0,0)$, $(1,0)$, and $(1,1)$ as a double iterated integral. \textbf{Do not evaluate the integral.}

\vspace*{8.5in}

\liner

\newpage\up

% (14.5) Parametrizing a Surface

\item Use spherical coordinates to give a parametrization corresponding to the portion of the surface \[z^2=x^2+y^2\] between the planes $z=1$ and $z=2$.

\vspace*{8.5in}

\liner

\newpage

% (14.5) Finding Surface Area

\item Use the cylindrical coordinate-based parametrization \[\vec{r}(\theta,z)=\left<2\cos\theta,2\sin\theta,z\right>\] to express the area of the surface $x^2+y^2=4$ between the planes $x=0$ and $x=2$ as a double iterated integral. \textbf{Do not evaluate the integral.}

\vspace*{8in}

\liner

\newpage\up

% (14.6) Surface Integral

\item Use the spherical coordinate-based parametrization \[\vec{r}(\phi,\theta)=\left<\sin\phi\cos\theta,\sin\phi\sin\theta,\cos\phi\right>\] to express the surface integral $\iint\limits_S 3z^2\,d\sigma$ as a double iterated integral of $\phi,\theta$, where $S$ is the upper half of the unit sphere $z=\sqrt{1-x^2-y^2}$. \textbf{Do not evaluate the integral.}

\vspace*{8in}

\liner

\newpage\up

%% Two pages of scratch work! %%
\centerline{Include extra scratch work below:}
\liner
\newpage
\centerline{Include extra scratch work below:}
\liner

\end{enumerate}

\end{document}
